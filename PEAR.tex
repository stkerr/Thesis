% Discuss the PEAR work and how it is relevant

\chapter{Physically Enhanced Authentication Ring}
\label{chapter:pear}

\section{Overview}
One problem that is present when using computers is that users typically are not aware of the security of the system
they are using. For instance, an attacker could have installed a key logger on a user's system to harvest every username
and password they have. Even with the best security systems on the machine in place, if the attacker is able to capture
a user's keystrokes, the other security is moot. 

A way to prevent this type of attack is by using an external device or an alternate channel to enter sensitive information,
such as passwords or credit card numbers. In this way, if a keylogger or the original system is compromised, the attacker
will not be able to recover those passwords, credit cards, or other sensitive information.

PEAR, or Physically Enhanced Authentication Ring, was designed to counteract this key logger threat to a system. In addition
to defending against keyloggers specifically, it increases security in general because it is the second part of a "two factor
authentication" system. It also is a physical system, specifically a peripheral physical system, since it incorporates its
own processing and interacts with the user's normal computer system. Thirdly, the PEAR system incorporates a PUF device,
so it is a good example of when PUF technology is useful.

From a high level perspective, a PEAR device is a device consisting of a PUF, a keypad, and some supporting circuitry. When
a user wishes to log on to a given service, rather than using the keyboard for a password, he enters a 4 digit PIN on the PEAR
device. The PEAR device then executes the PUF and then initiates a zero-knowledge proof of knowledge with the service
provider. Note that no sensitive data is actually input to the PC, which potentially has a keylogger. Any data that the PC
is requested to ferry between the PEAR device and the service provider is encrypted, so recording this data does not reveal
any information.

The typical use case is that a user requests a PEAR device from a service provider, such as a bank. The bank then configures
and mails the PEAR device to the user. The user sets his PIN number on the device and completes the enrollment protocol.
Then, when he desires to use the service, he requests the authentication protocol. He enters his PIN into the PEAR device
and the device then executes the rest of the authentication protocol. If successful, the service provider then allows
the user to access the service. 
Note that if a user already has a PEAR device from another service provider, he can easily use the same device for another
service; he simply must re-execute the enrollment procedure and enter a new PIN for the new service (or use the old PIN).

The system works by having every service provider associated with an ID number of some kind. Each user of the service will
also have an ID number associated with it. This allows both parties to identify themselves to each other.

\section{Protocol Details}
The PEAR system consists of two parts, an enrollment step, which is executed once initially and then an authentication step,
which is executed every time the user desires to use the service.
Table~\ref{tab:pearprotocol} presents a formalized description of the protocols, while Figure~\ref{fig:pearauthentication}
and Figure~\ref{fig:pearenrollment} give graphical representations of the different stages occurring.

As seen in the diagrams below, the four players in the PEAR system are the user himself, the PEAR device, the user's computer,
and the service provider. Note that the user's computer and the service provider are assumed to be connected over the Internet.

\begin{figure}[!ht]
\includegraphics[width=500px]{images/enrollment.jpg}
\label{fig:pearenrollment}
\caption{The enrollment stage of PEAR}
\end{figure}
\FloatBarrier

The user initially requests an enrollment procedure from the service provider. The service provider then sends the tuple
of $<Website ID, User Label>$ and a nonce to the PEAR device directly, over a secure channel. The user then enters his
new PIN number to the PEAR device. Once the PEAR device has these different pieces of information, it is able to execute
the steps contained inside the blue box in the diagram. All the pieces of information are hashed together. The resulting
hash is then used as input to the Password Generator and Verifier (PGV). Note that a PUF is an acceptable PGV. The results
from the PGV are then used in a commitment protocol. In addition to sending the results of the commitment to the service
provider, the device also sends the $<Website ID, User Label>$ tuple and a hash of the tuple combined with the committed value.

An interesting point to note is that during the enrollment stage, an "out of band" communication is required to deliver
the combination of the service provider's ID, the user's corresponding ID for that service, and a nonce value. This could
be done by installing these values on the hardware device before it is given to an end user. For instance, if PEAR was being
used with a bank, the bank might install these values before mailing the device to the user. If the user was adding a new
service to a PEAR device he already had, the tuple of $<Website ID, User Label>$ and the nonce might be delivered through
post or given over the phone to the user. The key point is that they are not delivered over the same channel as will be
used for the rest of the protocols (such as the Internet), since if an attacker is able to recover these values, he would
be able to make a fraudulent account.



\begin{figure}[!ht]
\includegraphics[width=500px]{images/auth.jpg}
\label{fig:pearauthentication}
\caption{The authentication stage of PEAR}
\end{figure}
\FloatBarrier

\begin{table}[!ht]
\label{tab:pearprotocol}
\caption{More formalized version of the PEAR protocols}
\noindent\makebox[\textwidth]{%
\begin{tabular}{|l|}
\hline
{\sf Enroll}($U$) - Device $T$ (using input data from user $U$) computes a commitment and enrolls the results with $S$. \\
\hline
- $C$ requests enrollment from $S$ \\
- $S$ sends the tuple $<$Label, ID$>$ and nonce $N$ to $T$ over a secure channel \\
- $U$ sends PIN to $T$ \\
- $T$ computes {\sf H}(ID, Label, PIN) as $H_{result}$ \\
- $T$ executes {\sf PGV}($H_{result}$) as $P_{result}$ \\
- $T$ sends {\sf Commit}($P_{result}$), $<$Label, ID$>$, {\sf H}({\sf Commit}($P_{result}$),Label,ID,$N$) to $S$, via $C$ \\
\hline
\hline
{\sf Authenticate}($U$) - Device $T$ (using input data from user $U$) authenticates itself as a registered user of $S$. \\
\hline
- $C$ initiates the authentication request from $S$ \\
- $S$ sends the tuple $<$Label, ID$>$ and {\sf Chal}($P_{result}$) to $T$ \\
- $U$ sends PIN to $T$ \\
- $T$ computes {\sf H}(ID, Label, PIN) as $H_{result}$ \\
- $T$ executes {\sf PGV}($H_{result}$) as $P_{result}$ \\
- $T$ responds with {\sf Prove}($P_{result}$), which $C$ forwards to $S$ \\
\hline
\end{tabular}
}
\end{table}
\FloatBarrier

\section{Security Considerations}

\section{Implementation}
From a high level view, Figure~\ref{fig:peararchitecture} describes the architecture of a PEAR enabled device.

\begin{figure}[!ht]
\includegraphics[width=500px]{images/pearimpl.jpg}
\label{fig:peararchitecture}
\caption{Implementation of a PEAR device}
\end{figure}
\FloatBarrier

\section{Acknowledgement}
This work was partially funded by Sypris Electronics. A paper on PEAR was published in 2010 in the SPRINGL 