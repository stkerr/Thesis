%
%  revised  front.tex  2011-09-02  Mark Senn  http://engineering.purdue.edu/~mark
%  created  front.tex  2003-06-02  Mark Senn  http://engineering.purdue.edu/~mark
%
%  This is ``front matter'' for the thesis.
%
%  Regarding ``References'' below:
%      KEY    MEANING
%      PU     ``A Manual for the Preparation of Graduate Theses'',
%             The Graduate School, Purdue University, 1996.
%      TCMOS  The Chicago Manual of Style, Edition 14.
%      WNNCD  Webster's Ninth New Collegiate Dictionary.
%
%  Lines marked with "%%" may need to be changed.
%


  % Dedication page is optional.
  % A name and often a message in tribute to a person or cause.
  % References: PU 15, WNNCD 332.
\begin{dedication}
  This is the dedication.
\end{dedication}

  % Acknowledgements page is optional but most theses include
  % a brief statement of apreciation or recognition of special
  % assistance.
  % Reference: PU 16.
\begin{acknowledgments}
A lot of people have had an impact in my life, both academic and personal. I wish to thank all of you 
that have influenced it in a positive way.

I would like to especially thank Professor Elisa Bertino. She took a risk on me as a freshmen and allowed 
me to get into research. I have greatly enjoyed learning and growing under her direction these past few 
years. Without such a great mentor, I don't think I would be where I am today.

Michael Kirkpatrick also deserves a lot of thanks. He helped guide some of my initial work with PUF devices,
pointing out things I missed and helping me mature as a researcher. His lessons were invaluable, not
only about the research topic themselves, but also about how research is done.

Finally, my parents deserve a special thanks. Without the wonderful upbringing and strong sense of values 
they instilled in me, none of this would have been possible. I love you Mom and Dad, thanks for pushing
me.

\end{acknowledgments}

  % The preface is optional.
  % References: PU 16, TCMOS 1.49, WNNCD 927.
% \begin{preface}
%  This is the preface.
% \end{preface}

  % The Table of Contents is required.
  % The Table of Contents will be automatically created for you
  % using information you supply in
  %     \chapter
  %     \section
  %     \subsection
  %     \subsubsection
  % commands.
  % Reference: PU 16.
\tableofcontents

  % If your thesis has tables, a list of tables is required.
  % The List of Tables will be automatically created for you using
  % information you supply in
  %     \begin{table} ... \end{table}
  % environments.
  % Reference: PU 16.
\listoftables

  % If your thesis has figures, a list of figures is required.
  % The List of Figures will be automatically created for you using
  % information you supply in
  %     \begin{figure} ... \end{figure}
  % environments.
  % Reference: PU 16.
\listoffigures

  % List of Symbols is optional.
  % Reference: PU 17.
%\begin{symbols}
%  $m$& mass\cr
%  $v$& velocity\cr
%\end{symbols}

  % List of Abbreviations is optional.
  % Reference: PU 17.
\begin{abbreviations}
  PUF& Physically Unclonable Function\cr
  PEAR& Physically Enhanced Authentication Ring\cr
  ROK& Read Once Key\cr
  RS& Reed Solomon Codes\cr
  ZKPK& Zero Knowledge Proof of Knowledge
\end{abbreviations}

  % Nomenclature is optional.
  % Reference: PU 17.
%\begin{nomenclature}
%  Alanine& 2-Aminopropanoic acid\cr
%\end{nomenclature}

  % Glossary is optional
  % Reference: PU 17.
%\begin{glossary}
%	Physical System& A system that interacts with the physical world or has some sort of hardware component to it (e.g. not a pure software implementation)\cr
%  chick& female, usually young\cr
%  dude& male, usually young\cr
%\end{glossary}

  % Abstract is required.
  % Note that the information for the first paragraph of the output
  % doesn't need to be input here...it is put in automatically from
  % information you supplied earlier using \title, \author, \degree,
  % and \majorprof.
  % Reference: PU 17.
\begin{abstract}
Physical systems are systems that interact with the physical world around them. They are becoming increasingly
computationally powerful as faster microprocessors are installed. This allows for many more types of applications
and functionality to be implemented. However, this also means that there is a greater security risk. Much of this
risk has to do with confirming the device as an authentic device or not. This goal is achievable using a technology
known as Physically Unclonable Functions (PUFs). PUFs use the intrinsic differences in hardware behavior to produce a 
random function that is unique to that hardware instance. When combined with existing cryptographic techniques,
these PUFs enable many different types of applications, such
as read once keys, secure communications, and securing smart grids.
\end{abstract}
