% Discuss the nature of physical system and some of the unique problems associated with them

\chapter{Physical Systems}
\label{chapter:physicalsystems}

Physical systems present an interesting problem domain for study. In contrast to software systems,
they are subjected to multiple different factors that all require consideration during design. Physical
systems frequently must be able to cope with environmental factors such as temperature change, moisture,
or questionable power systems. 

A purely software system may be able to assume it will only receive input from a standard input and output
channel. In contrast, a physical system must be able
to account for multiple different input sources, especially input types that might not have been intended. 
A physical system might receive input directly from end users, networking devices, sensors. A physical
system could also consider environment changes as a sort of secondary, unintended input. For example,
the device's power may fluctuate, potentially changing the behaviour of internal circuits. A temperature change
could cause the sensitivity of a certain component to increase or decrease, which will in turn alter the behaviour
of the system. These are but a few examples of the various factors that a physical system must be properly
designed to handle and account for.

% Describe some of the typical use cases of physical systems
\section{Typical Organization and Use Case of Physical Systems}
Physical systems are a very broad category that covers a large set of devices, applications, and use cases.
Because of this, it is difficult to discuss physical systems generically. Rather, this section details some
of the common configurations that physical systems take. The rest of the work will regard physical systems as
belonging to one of the configurations discussed.

Despite it being difficult to characterize physical systems in general terms, their operation can be viewed using the mathematical
relation below. Physical inputs are inputs from physical interfaces, such as buttons, control terminals, or radio signals. Other
inputs might be information received over the network or from some sort of attached peripheral.

\begin{align*}
Output = System_{Physical}(Physical Inputs, Other Inputs, Environment)
\end{align*}

The three configurations of physical systems that will be considered are that of the standalone, deployed, or peripheral physical
systems. Each is distinct based upon how much interaction it has, not only with the physical world, but also with other physical
systems or remote devices. Peripheral systems have the most interaction with remote parties and other systems while standalone
systems have the least. In terms of the equation above, the three categories vary based on what sort of 'Other Inputs' are passed
to them.

% Standalone device (in the field)
\subsection*{Standalone Physical Systems}
One common configuration of a physical system is that of a standalone physical system. This means that the
physical system is not reliant on communicating with another physical system; it is deployed and functions
independently. An example of this could be a garage door opener. There might be a control pad on the side of the
garage which can open or close the garage door. Additionally, there could be an option to open the door remotely
using some sort of radio frequency device.

Standalone physical systems are more straightforward to deal with in a lot of cases. The garage opener example has
a very specific use case, defined inputs (control pad and remote control) and defined outputs (open or close the garage
door). These qualities typically do not change nor are updated often, if ever. As such, it is typically easy to create a sort
of state diagram to model the behaviour of standalone physical systems.

% Device in the field that communicates back
\subsection*{Deployed Physical Systems}
Another category of physical systems is that of the 'deployed physical system'. This is a type of physical system that not only
interacts with the environment it is in, but may also communicate with another physical system or some sort of remote
server. A cash register is a good example of a deployed physical system. It takes input from cashiers, who can record transactions,
print receipts, and insert or remove currency from it. However, it also communicates with remote servers in certain cases, such
as when a credit card is used. It must interact with the physical environment, but also must interact with remote servers to verify
the credit card transactions.

Because a deployed physical system must potentially interact with a remote party, it is more complex than a standalone physical
system. It must contend with the same sorts of issues that standalone systems do, but also has to deal with issues that could
relate to the remote communication or other physical system.  As such, it is more complex and difficult to model a deployed
physical system than a standalone physical system.

% Physical peripheral
\subsection*{Peripheral Physical Systems}
Peripheral physical systems are the most complex type of physical system. These are typically normally called 'peripherals'. That
is, they do not provide the main functionality of a system, but augment it's ability in some way. An example could be a 
programmable sensor array. The sensor array could be connected to a network through which it receives commands. The array
would then take sensor readings and communicate them back over the network. Not only does the array have to 
interact with the physical environment to take readings, but there is also the component of dealing with the
command and control element from the network connection.

Peripheral physical systems are characterized by the fact that they not only require interaction with the physical world, but
also with other physical systems or with a remote connection. Because of this, it is very difficult to model the system, since
there are a very large number of ways that the other communicating party could potentially behave, in addition to any difficulties
involved with modelling the physical inputs themselves. 



% Describe the coming sections of the paper

\section{Benign Considerations}
There are multiple ways that a physical system or device could fail. These failures may not necessarily have
any malicious reason behind them, but they still must be considered and evaluated during system design.

% Discuss the benign difficulties associated with physical systems

\subsection{Device Failure}
% Device failure

\subsection{Device Degradation}
% Device degradation 

\section{Malicious Considerations}
% Discuss different malicious problems with physical systems

% DOS
% MITM
% Impersonation
