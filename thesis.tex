%
%  thesis.tex  2011-07-01  Mark Senn  http://engineering.purdue.edu/~mark
%
%  This is the thesis ``root file''.
%
%  To print the final copy of your thesis put a '%'
%  in front of the \includeonly command and type
%  (from page 71 of _LaTeX User's Guide and Reference Manual_, 2nd edition):
%      latex thesis
%      bibtex thesis
%      latex thesis
%      latex thesis
%
%  In "Reference:" listings below:
%      KEY  MEANING
%      TM   ``A Manual for the Preparation of Graduate Theses'',
%             seventh revised edition, The Graduate School, 2006.
%             http://www2.itap.purdue.edu/gradschool//Publications/graduate-thesis-manual.pdf
%      PU   ``A Manual for the Preparation of Graduate Theses'',
%           The Graduate School, Purdue University, 1996.
%           http://www2.itap.purdue.edu/gradschool//Publications/graduate-thesis-manual.pdf
%
%  Search for "CHANGE" below and change things as necessary.
%  I recommend putting "%%" before any existing lines that
%  need to be changed and adding your new line(s) immediately
%  below the existing lines.
%

% See http://www.ecn.purdue.edu/~mark/puthesis/#Options
% for documentclass options.
% CHANGE NEXT LINE?
\documentclass[cs,thesis]{puthesis}

% Define "align" environment used in demo-mathematics.tex.
% CHANGE NEXT LINE?
\usepackage{amsmath}

% Define "multicols" environment environment used in demo-multicols.tex.
% CHANGE NEXT LINE?
\usepackage{multicol}

% Define "subfigure" environment used in "demo-figure.tex".
% CHANGE NEXT LINE?
\usepackage{subfigure}

% Title of thesis (used on cover and in abstract).
% The title shown must be the full, official title of the
% thesis.  Superscripts and subscripts are not permitted in
% the title.
% Reference: TM 26.
% Use \title{Put Title Here} for a one-line title.
% Use \\ to separate lines.
% Put % at the end of the last line to avoid getting an extra space
% in the abstract.
% There are two forms of title: one line or more than one line.
% There are examples of both below.
% Only use one \title.
% CHANGE NEXT FOUR LINES.
\title{Secure Physical System Design Leveraging PUF Technology}

% First author name with first name first is used for cover.
% Second author name with last name first is used for abstract.
% Your full name as it appears in the University records appears
% on the cover.
% Reference: TM 26, 29.
% There are two forms of author, with and without initials.
% There are examples of both below.
% Only use one \author line.
% CHANGE NEXT TWO LINES.
\author{Samuel Kerr}{Kerr, Samuel}

% First is long title of degree (used on cover).
% Second is abbreviation for degree (used in abstract).
% Third is the month the degree was (will be) awarded (used on cover
% and abstract).
% Last is the year the degree was (wlll be) awarded (used on cover
% and abstract).
% The degree title for all doctoral candidates is ``Doctor of Philosophy.''
% The precise degree names for master's candidates appear in the list of
% ``Degrees Offered'' in the Graduate School bulletin.
% The date is the month and year that the degree is actually awarded.
% (If you have registered for ``degree only,'' revise the thesis title
% page to reflect the new date on which the degree is to be awarded.)
% Reference: TM 26--27, 30.
% CHANGE NEXT LINE?
\pudegree{Master of Science}{M.S.}{May}{2012}

% Major professor (used in abstract).
% Use, for example:
%     \majorprof{John Q. Professor}
%     \majorprofs{John Q. Professor and Thomas R. Jones}
%     \majorprofs{John Q. Professor, Thomas R. Jones, and David S. Smith}
% depending on the number of major professors you have.
% CHANGE NEXT LINE.
\majorprof{Elisa Bertino}

% Campus (used only on cover)
% Use one of the following:
%     Fort Wayne
%     Hammond
%     Indianapolis
%     West Lafayette
%     Westville
% Reference: TM 27.
% CHANGE NEXT LINE?
\campus{West Lafayette}

% My command definitions not specific to my thesis.
% CHANGE NEXT LINE?
%
%  mydefs.tex  2007-03-19  Mark Senn  http://www.ecn.purdue.edu/~mark
%
%  Command definitions that can be used in all documents that have
%      %
%  mydefs.tex  2007-03-19  Mark Senn  http://www.ecn.purdue.edu/~mark
%
%  Command definitions that can be used in all documents that have
%      %
%  mydefs.tex  2007-03-19  Mark Senn  http://www.ecn.purdue.edu/~mark
%
%  Command definitions that can be used in all documents that have
%      \input{mydefs}
%

% CHANGE NEXT 3 LINES?
% Define \be and \ee to start and end the equation environment.
\newcommand{\be}{\begin{equation}}
\newcommand{\ee}{\end{equation}}

% CHANGE NEXT 12 LINES?
% Define \Repeat so, for example,
%     \Repeat{whatever}{10}
% is the same as typing whatever 10 times.
\newcount{\myi}
\newcommand{\Repeat}[2]{%
    \myi=0
    \loop
        \ifnum\myi<#2
        #1
        \advance\myi by 1
    \repeat
}

% CHANGE NEXT 3 LINES?
% Make "\Sum ab" or "\Sum{a}{b}" do "\sum_{a}^{b}".
% This can only be used when in math mode.
\newcommand\Sum[2]{\sum_{#1}^{#2}}

% CHANGE NEXT 4 LINES?
% Make "\xn" do "$x_n$".
% Because this definition contains the "$" to go into math mode
% this definition must be used when not in math mode.
\newcommand{\xn}{$x_n$}

% CHANGE NEXT 5 LINES?
% Since \xn is already defined we must use \renewcommand to redefine it.
% Normally you would not have the above definition for \xn in this file
% if you were just going to override it later.
% The \ensuremath goes into math mode if not already in math mode.
\renewcommand{\xn}{\ensuremath{x_n}}


%

% CHANGE NEXT 3 LINES?
% Define \be and \ee to start and end the equation environment.
\newcommand{\be}{\begin{equation}}
\newcommand{\ee}{\end{equation}}

% CHANGE NEXT 12 LINES?
% Define \Repeat so, for example,
%     \Repeat{whatever}{10}
% is the same as typing whatever 10 times.
\newcount{\myi}
\newcommand{\Repeat}[2]{%
    \myi=0
    \loop
        \ifnum\myi<#2
        #1
        \advance\myi by 1
    \repeat
}

% CHANGE NEXT 3 LINES?
% Make "\Sum ab" or "\Sum{a}{b}" do "\sum_{a}^{b}".
% This can only be used when in math mode.
\newcommand\Sum[2]{\sum_{#1}^{#2}}

% CHANGE NEXT 4 LINES?
% Make "\xn" do "$x_n$".
% Because this definition contains the "$" to go into math mode
% this definition must be used when not in math mode.
\newcommand{\xn}{$x_n$}

% CHANGE NEXT 5 LINES?
% Since \xn is already defined we must use \renewcommand to redefine it.
% Normally you would not have the above definition for \xn in this file
% if you were just going to override it later.
% The \ensuremath goes into math mode if not already in math mode.
\renewcommand{\xn}{\ensuremath{x_n}}


%

% CHANGE NEXT 3 LINES?
% Define \be and \ee to start and end the equation environment.
\newcommand{\be}{\begin{equation}}
\newcommand{\ee}{\end{equation}}

% CHANGE NEXT 12 LINES?
% Define \Repeat so, for example,
%     \Repeat{whatever}{10}
% is the same as typing whatever 10 times.
\newcount{\myi}
\newcommand{\Repeat}[2]{%
    \myi=0
    \loop
        \ifnum\myi<#2
        #1
        \advance\myi by 1
    \repeat
}

% CHANGE NEXT 3 LINES?
% Make "\Sum ab" or "\Sum{a}{b}" do "\sum_{a}^{b}".
% This can only be used when in math mode.
\newcommand\Sum[2]{\sum_{#1}^{#2}}

% CHANGE NEXT 4 LINES?
% Make "\xn" do "$x_n$".
% Because this definition contains the "$" to go into math mode
% this definition must be used when not in math mode.
\newcommand{\xn}{$x_n$}

% CHANGE NEXT 5 LINES?
% Since \xn is already defined we must use \renewcommand to redefine it.
% Normally you would not have the above definition for \xn in this file
% if you were just going to override it later.
% The \ensuremath goes into math mode if not already in math mode.
\renewcommand{\xn}{\ensuremath{x_n}}




% My command definitions specific to my thesis.

% CHANGE NEXT LINE TWO LINES?
% Set things up so \margins will show where the margins on the page are.
\newcommand{\margins}{\Repeat{Show where the margins for the page are.}{4}}

% CHANGE NEXT TWO LINES?
% Let typing "\en" be exactly the same as typing "\ensuremath". 
\let\en=\ensuremath

% CHANGE NEXT FIVE LINES?
% Define a \ve command with two arguments, so if it called with
%     \ve an
% it will expand to
%     {\en{a_1},~\en{a_2},\ \ldots,~\en{a_{n}}}
\newcommand{\ve}[2]{\en{#1_1},~\en{#1_2},\ \ldots,~\en{#1_{#2}}}


% To LaTeX only some parts of your thesis put the
% names of the parts to include here.  For example,
% \includeonly{front} would only process front.tex.
% \includeonly{front,introduction} would only process
% front.tex and introduction.tex.
% To print the final copy of your thesis put a '%'
% in front of the \includeonly command and run LaTeX
% three times to make sure that all cross-references
% are correct.  Then run BibTeX once and LaTeX twice
% more.
% CHANGE NEXT LINE?
\includeonly{front,introduction}

\begin{document}

% Start a new volume for your thesis.  All theses must have at least one
% volume.  If your thesis is too long to fit in one binder put another
% "\volume" between chapters below.
\volume

% Front matter (dedication, etc.).
%
%  revised  front.tex  2011-09-02  Mark Senn  http://engineering.purdue.edu/~mark
%  created  front.tex  2003-06-02  Mark Senn  http://engineering.purdue.edu/~mark
%
%  This is ``front matter'' for the thesis.
%
%  Regarding ``References'' below:
%      KEY    MEANING
%      PU     ``A Manual for the Preparation of Graduate Theses'',
%             The Graduate School, Purdue University, 1996.
%      TCMOS  The Chicago Manual of Style, Edition 14.
%      WNNCD  Webster's Ninth New Collegiate Dictionary.
%
%  Lines marked with "%%" may need to be changed.
%

  % Dedication page is optional.
  % A name and often a message in tribute to a person or cause.
  % References: PU 15, WNNCD 332.
\begin{dedication}
  This is the dedication.
\end{dedication}

  % Acknowledgements page is optional but most theses include
  % a brief statement of apreciation or recognition of special
  % assistance.
  % Reference: PU 16.
\begin{acknowledgments}
  This is the acknowledgments.
\end{acknowledgments}

  % The preface is optional.
  % References: PU 16, TCMOS 1.49, WNNCD 927.
\begin{preface}
  This is the preface.
\end{preface}

  % The Table of Contents is required.
  % The Table of Contents will be automatically created for you
  % using information you supply in
  %     \chapter
  %     \section
  %     \subsection
  %     \subsubsection
  % commands.
  % Reference: PU 16.
\tableofcontents

  % If your thesis has tables, a list of tables is required.
  % The List of Tables will be automatically created for you using
  % information you supply in
  %     \begin{table} ... \end{table}
  % environments.
  % Reference: PU 16.
\listoftables

  % If your thesis has figures, a list of figures is required.
  % The List of Figures will be automatically created for you using
  % information you supply in
  %     \begin{figure} ... \end{figure}
  % environments.
  % Reference: PU 16.
\listoffigures

  % List of Symbols is optional.
  % Reference: PU 17.
%\begin{symbols}
%  $m$& mass\cr
%  $v$& velocity\cr
%\end{symbols}

  % List of Abbreviations is optional.
  % Reference: PU 17.
\begin{abbreviations}
  PUF& Physically Unclonable Function\cr
  PEAR& Physically Enhanced Authentication Ring\cr
  ROK& Read Once Key\cr
\end{abbreviations}

  % Nomenclature is optional.
  % Reference: PU 17.
%\begin{nomenclature}
%  Alanine& 2-Aminopropanoic acid\cr
%\end{nomenclature}

  % Glossary is optional
  % Reference: PU 17.
\begin{glossary}
	Physical System& A system that interacts with the physical world or has some sort of hardware component to it (e.g. not a pure software implementation)\cr
%  chick& female, usually young\cr
%  dude& male, usually young\cr
\end{glossary}

  % Abstract is required.
  % Note that the information for the first paragraph of the output
  % doesn't need to be input here...it is put in automatically from
  % information you supplied earlier using \title, \author, \degree,
  % and \majorprof.
  % Reference: PU 17.
\begin{abstract}
  This is the abstract.
\end{abstract}


% Put chapter \include commands here.
% CHANGE \include{...} COMMANDS BELOW?
%
%  revised  introduction.tex  2011-09-02  Mark Senn  http://engineering.purdue.edu/~mark
%  created  introduction.tex  2002-06-03  Mark Senn  http://engineering.purdue.edu/~mark
%
%  This is the introduction chapter for a simple, example thesis.
%


\chapter{Introduction}

This is the introduction.
The first paragraph after a heading is not indented.

This is a sentence.
This is a sentence.
This is a sentence.
This is a sentence.
This is a sentence.


\section{Section Heading}

This is a sentence.
This is a sentence.
This is a sentence.
This is a sentence.
This is a sentence.


\subsection{Subsection heading}

This is a sentence.
This is a sentence.
This is a sentence.
This is a sentence.
This is a sentence.


\subsubsection{Subsubsection heading}

This is a sentence.
This is a sentence.
This is a sentence.
This is a sentence.
This is a sentence.


% Summary and/or conclusions are optional but often used.
% The summary and/or conclusions often are the last
% major division(s) of the text.
% Reference: TM 32.
% CHANGE NEXT LINE?
%
%  summary.tex  2007-02-06  Mark Senn  http://www.ecn.purdue.edu/~mark
%

\chapter{Summary}
\label{chapter:conclusion}

Systems that interact with the physical world are not only becoming more pervasive, but also more powerful
from a computing perspective. It is helpful to classify these systems into groups of standalone, deployed, and
peripheral physical systems. The distinction lies in how they communicate with other systems,
such as the ones connected over the internet.

There are many different security issues facing physical systems. Not only must they compensate for threats similar
to those faced in software, such as impersonations, man-in-the-middles, and replay attacks, they must also
contend with threats specific to a system that exists in the physical world. These include power analysis attacks,
which can glean information from power consumption, signal injection attacks, which can alter system behavior by
bombarding the system with wireless signals, and simple tampering, such as breaking components or attaching logic
probes.

The technology of Physically Unclonable Function (PUF) was presented, which allows strong guarantees of device authenticity
to be made by leveraging the challenge-response properties of PUF. These devices are able to offer these guarantees
since they cannot be duplicated using a manufacturing process, so the responses they give to given challenges are
always distinct from other PUFs. PUFs require a certain amount of support circuitry to deal with bit errors that occasionally
occur, but this is acceptable and solvable using existing error correction techniques.

It is possible to utilize PUF devices in conjunction with certain cryptographic protocols, such as zero knowledge proof
of knowledge (ZKPK) proofs, to implement interesting applications. Several different applications were presented, each of
which demonstrated the use of PUF in a different context.

The PUF ROK application leveraged a PUF device to create keys that, once used, are unrecoverable. This was done by
giving the PUF an initial ``seed" value and then creating a feedback loop with the PUF. When the PUF generated a response,
it would overwrite the previously stored value. Since PUFs are one-way functions, there is no way to go backwards and
recover the value that was previously used. These ROKs could then be used to create ``self-destructing" documents
or for an authority to give a delegate a limited number of a higher privilege level.

The PEAR application used PUFs in a way to uniquely and securely identify devices, despite insecure communication
channels. In addition to a PUF, a ZKPK protocol was used to provide this benefit. The PUF as well as an external keypad
make up a PEAR device so that data can be entered securely. The initial goal was to allow users to be able to log in
to websites securely, even if a hardware key logger was attached to the keyboard. This is possible, but PEAR also
allows this capability in the presence of software threats on the PC, since all traffic is encrypted.

Finally, the smart grid project utilized PUFs to provide strong guarantees of smart meters' identity. This is critical
because if the utility company was not sure of meters it was communicating with, catastrophic attacks would be possible,
such as overloading of power handling circuits. The PUF was again used in conjunction with ZKPK proofs to
protect information in transit between the two parties. Additionally, since the utility company is required to maintain
a large database, storing ZKPK commitments, rather than secrets directly, is a much more secure approach. Another
interesting point of this application was that the PUF generated and maintained a master key internally, 
but derived keys were used in all steps of the protocol. This protects the long term security of the device, even in the 
event of certain security compromises.

These different applications show the versatility of PUF technology in helping to secure physical systems.
Depending on which cryptographic tools and protocols they are used in conjunction with, PUFs can be used
in a variety of different ways, as was demonstrated.

% Recommendations are optional.
% You may include recommendations as a major division if your
% subject matter and research dictate.
% Reference: TM 32.
% CHANGE NEXT LINE?
%
%  recommendations.tex  2007-02-06  Mark Senn  http://www.ecn.purdue.edu/~mark
%

\chapter{Recommendations}

Buy low.  Sell high.


% Appendices are optional.
% Appendices are not necessarily part of every thesis. Appendices are used
% for supplementary illustrative material, original data, computer programs,
% and other material not necessarily appropriate for inclusion within the
% text of your thesis. 
% Reference: TM 33.
% Use "\appendix" for one appendix or "\appendices" for more than one
% appendix.
% CHANGE NEXT 7 LINES?
\appendices
%
%  revised  demo-citations.tex  2011-09-02  Mark Senn  http://www.ecn.purdue.edu/~mark
%  created  demo-citations.tex  2007-03-21  Mark Senn  http://www.ecn.purdue.edu/~mark
%


\chapter{Demonstrate Citations}

I typed

\begin{verbatim}
    For \LaTeX\ answers I refer to
    % note to self: {\em \LaTeX: A Document Preparation System\/}
    \cite{Lamport:1994}
    and then to
    % note to self: {\em The \LaTeX\ Companion\/}
    \cite{Goossens:1994}
    or
    % note to self: {\em A Guide to LaTeX\/} (1999)
    \cite{Kopka:1999}.
    % note to self: {\em A Guide to LaTeX\/} (1999)
    \cite{Kopka:1999}
    is an updated edition of the 1995 edition
    \cite{Kopka:1995}.
\end{verbatim}

to get

\begin{quotation}
    For \LaTeX\ answers I refer to
    % note to self: {\em \LaTeX: A Document Preparation System\/}
    \cite{Lamport:1994}
    and then to
    % note to self: {\em The \LaTeX\ Companion\/}
    \cite{Goossens:1994}
    or
    % note to self: {\em A Guide to LaTeX\/} (1999)
    \cite{Kopka:1999}.
    % note to self: {\em A Guide to LaTeX\/} (1999)
    \cite{Kopka:1999}
    is an updated edition of the 1995 edition
    \cite{Kopka:1995}.
\end{quotation}

%
%  demo-figures.tex  2009-10-30  Mark Senn  http://engineering.purdue.edu/~mark
%
%  Demonstrate how to do figures.
%

\chapter{Demonstrate Figures}

The
\verb+h+
specifier used in all the examples below
tells \LaTeX\ to put the figure
``here''
instead of trying
to find a good spot
at the top or bottom of a page.
Specifiers can be combined, for example,
``\verb+\begin{figure}[htbp!]+''.

The complete list of specifiers:

\begin{center}
    \renewcommand{\baselinestretch}{1}\normalsize
    \begin{tabular}{ll}
        \bf Specifier& \bf Description\cr
        \tt b& bottom of page\cr
        \tt h& here on page\cr
        \tt p& on separate page of figures\cr
        \tt t& top of page\cr
        \tt !& try hard to put figure as early as possible\cr
    \end{tabular}
\end{center}

Label ``fi:not-centered'' is ``\ref{fi:not-centered}''.
Label ``sf:four-parts-c'' is ``\ref{sf:four-parts-c}''.

\Repeat{This is the first paragraph.}{5}

\begin{figure}[h]
  \includegraphics{plot.eps}
  \caption{%
    By default figures are not centered.
    This is a long caption to demonstrate that captions are single spaced.
  }
  \label{fi:not-centered}
\end{figure}

\Repeat{This is the second paragraph.}{10}

\begin{figure}[h]
  \centering
  \includegraphics{plot.eps}
  \caption{Use {\tt \char'134centering\/} to center figures.}
  \label{fi:centered}
\end{figure}

\Repeat{This is the third paragraph.}{15}

\begin{figure}[h]
  \centering
  \includegraphics{plot.eps}
  \caption{This is another figuure.}
  \label{fi:another}
\end{figure}

\Repeat{This is the fourth paragraph.}{10}

\begin{figure}[h]
  \centering 
  \subfigure[First subcaption.]{\label{sf:two-parts-a}  \includegraphics[width=0.3\textwidth]{plot.eps}}%
  \hskip 0.5truein
  \subfigure[Second subcaption.]{\label{sf:two-parts-b}\includegraphics[width=0.3\textwidth]{plot.eps}}
  \caption{This figure has two parts.}
  \label{fi:two-parts}
\end{figure}

\Repeat{This is the fifth paragraph.}{10}

\begin{figure}[h]
  \centering
  \subfigure[First subcaption.]{\label{sf:four-parts-a}  \includegraphics[width=0.3\textwidth]{plot.eps}}%
  \hskip 0.5truein
  \subfigure[Second subcaption.]{\label{sf:four-parts-b}\includegraphics[width=0.3\textwidth]{plot.eps}}
  \subfigure[Third subcaption.]{\label{sf:four-parts-c}\includegraphics[width=0.3\textwidth]{plot.eps}}%
  \hskip 0.5truein
  \subfigure[Fourth subcaption.]{\label{sf:four-parts-d}\includegraphics[width=0.3\textwidth]{plot.eps}}
  \caption{This figure has four parts.}
  \label{fi:four-parts}
\end{figure}

\Repeat{This is the sixth paragraph.}{10}

%
%  THIS FILE DOES SOME UNUSUAL THINGS TO MAKE
%  IT EASIER TO DO DEMONSTRATIONS.  IT SHOULD
%  NOT BE USED AS AN EXAMPLE OF HOW TO PREPARE
%  A FILE.  SEE THE OUTPUT OF THIS FOR LATEX
%  INPUT AND OUTPUT EXAMPLES.
%




%
%  demo-mathematics.tex  2008-12-09  Mark Senn  http://engineering.purdue.edu/~mark
%

\chapter{Demonstrate Mathematics}

    % Use single spacing.
    \Baselinestretch{1}

    % You don't normally need this.
    \mbox{}

    \begin{verbatim}
% From _More Math Into LaTeX_, 4th Edition, page 152:
%     TeX uses $$ to open and close a displayed math environment.
%     In LaTeX, this may occassionally cause problems.  Don't do it.
\[
    E = mc^2
\]
    \end{verbatim}
% From _More Math Into LaTeX_, 4th Edition, page 152:
%     TeX uses $$ to open and close a displayed math environment.
%     In LaTeX, this may occassionally cause problems.  Don't do it.
\[
    E = mc^2
\]
    \vskip\baselineskip
    \hrule
    \vskip0.5\baselineskip
    \filbreak

    \begin{verbatim}
\begin{equation}
    E = mc^2
\end{equation}
    \end{verbatim}
\begin{equation}
    E = mc^2
\end{equation}
    \vskip\baselineskip
    \hrule
    \vskip0.5\baselineskip
    \filbreak

    \begin{verbatim}
% Mydefs.tex defines \be to be \begin{equation} and
% \ee to be \end{equation}.
\be
    E = mc^2
\ee
    \end{verbatim}
% Mydefs.tex defines \be to be \begin{equation} and
% \ee to be \end{equation}.
\be
    E = mc^2
\ee
    \vskip\baselineskip
    \hrule
    \vskip0.5\baselineskip
    \filbreak

    \begin{verbatim}
\be
    x = -\frac{b}{2a} \pm \frac{\sqrt{b^2 - 4ac}}{2a}
\ee
    \end{verbatim}
\be
    x = -\frac{b}{2a} \pm \frac{\sqrt{b^2 - 4ac}}{2a}
\ee
    \vskip\baselineskip
    \hrule
    \vskip0.5\baselineskip
    \filbreak

    \begin{verbatim}
% requires \usepackage{amsmath}; use align* for no equation number
\begin{align}
    a = {}& b + c\\
    x = {}& y + z
\end{align}
    \end{verbatim}
% requires \usepackage{amsmath}; use align* for no equation number
\begin{align}
    a = {}& b + c\\
    x = {}& y + z
\end{align}
    \vskip\baselineskip
    \hrule
    \vskip0.5\baselineskip
    \filbreak

    \begin{verbatim}
\[
    Z = \left(
        \begin{array}{cc}
            a& b\\
            c& d
        \end{array}
    \right)
\]
    \end{verbatim}
\[
    Z = \left(
        \begin{array}{cc}
            a& b\\
            c& d
        \end{array}
    \right)
\]
    \vskip\baselineskip
    \hrule
    \vskip0.5\baselineskip
    \filbreak

    \begin{verbatim}
\begin{equation}
    \begin{split}
        a = {}& b + c\\
            {}& + d + e
    \end{split}      
\end{equation}
    \end{verbatim}
\begin{equation}
    \begin{split}
        a = {}& b + c\\
            {}& + d + e
    \end{split}      
\end{equation}
    \vskip\baselineskip
    \hrule
    \vskip0.5\baselineskip
    \filbreak

    \begin{verbatim}
\be
    (\cos x)^2 + (\sin x)^2 = 1
\ee
    \end{verbatim}
\be
    (\cos x)^2 + (\sin x)^2 = 1
\ee
    \vskip\baselineskip
    \hrule
    \vskip0.5\baselineskip
    \filbreak

    \begin{verbatim}
If $X = \cos x$ and $Y = \sin x$ then $X^2 + Y^2 = 1$.
    \end{verbatim}
If $X = \cos x$ and $Y = \sin x$ then $X^2 + Y^2 = 1$.
    \vskip\baselineskip
    \hrule
    \vskip0.5\baselineskip
    \filbreak

%
%  demo-multicols.tex  2007-03-19  Mark Senn  http://www.ecn.purdue.edu/~mark
%
%  Demonstrate multicols.
%
%  The multicols package must be loaded for this to work.
%  To load the multicols package put
%      \usepackage{multicols}
%  between the "\documentclass" and "\begin{document}" commands.
%

\chapter{Demonstrate Multicols}

% Put this amount of space between the columns.
\setlength{\columnsep}{0.5truein}

% Separate the columns with a vertical rule this wide.
\setlength{\columnseprule}{0.4pt}

\Repeat{This is one column.}{25}

\begin{multicols}{2}
\Repeat{This is two columns.}{25}
\end{multicols}

\begin{multicols}{3}
\Repeat{This is three columns.}{25}
\end{multicols}

\begin{multicols}{4}
\Repeat{This is four columns.}{25}
\end{multicols}

\begin{multicols}{5}
\Repeat{This is five columns.}{25}
\end{multicols}

%
%  demo-tables.tex  2009-09-29  Mark Senn  http://engineering.purdue.edu/~mark
%
%  Demonstrate how to do tables.
%

\chapter{Demonstrate Tables}

\begin{tabular}{ll}
    \bf Label& \bf Number\\
    ta:text-only& \ref{ta:text-only}\\
    ta:fruit&     \ref{ta:fruit}
\end{tabular}

\newlength{\ta}
\newlength{\tb}
\newlength{\tc}

\settowidth{\ta}{\vbox{\hbox{Money}\hbox{Market}}}
\settowidth{\tb}{\vbox{\hbox{Stocks}\hbox{and}\hbox{Bonds}}}
\settowidth{\tc}{\vbox{\hbox{Money}\hbox{Market}\hbox{and}\hbox{Stocks}}}

%{\renewcommand{\baselinestretch}{1}
%  \begin{table}
%    \caption{%
%      \hfil Allocation of the IRA and Keogh Wealth\hfil\break
%      \mbox{}\hfil for Investors With or Without Brokerage Accounts\hfil
%    }
%    \label{tab:ira}
%    \begin{center}
%      \begin{tabular}%
%        {%
%          |%
%          c%
%          |%
%          >{\centering\hspace{0pt}}m{\the\ta}%  Money Market
%          |%
%          c%                                    Stocks 
%          |%
%          c%                                    Bonds
%          |%
%          c%                                    Diversified
%          |%
%          >{\centering\hspace{0pt}}m{\the\tb}%  Stocks and Bonds
%          |%
%          >{\centering\hspace{0pt}}m{\the\tc}%  Money Market and Stocks
%          |%
%          c%                                    Others
%          |%
%        }
%        \hline
%        IMP&
%          Money Market&
%          Stocks&
%          Bonds&
%          Diversified&
%          Stocks and Bonds&
%          Money Market and Stocks&
%          Others\tabularnewline
%        \hline
%        1& 14.19\%& 57.71\%& 12.21\%& 4.50\%& 7.36\%& 3.04\%& 0.99\%\tabularnewline \hline
%        2& 14.08\%& 58.18\%& 12.32\%& 4.44\%& 7.30\%& 2.80\%& 0.88\%\tabularnewline \hline
%        3 &14.26\%& 58.09\%& 12.27\%& 4.50\%& 7.19\%& 2.75\%& 0.94\%\tabularnewline \hline
%        4 &13.94\%& 58.11\%& 12.14\%& 4.78\%& 7.35\%& 2.68\%& 0.99\%\tabularnewline \hline
%        5 &13.92\%& 58.13\%& 11.93\%& 4.56\%& 7.60\%& 2.98\%& 0.88\%\tabularnewline \hline
%      \end{tabular}
%    \end{center}
%    This table presents the allocations of the wealth in the IRA
%    and Keogh accounts in various asset classes.
%    Results from each set of imputed data are presented here.
%    The first column lists the number of the imputations,
%    and rest of the columns lists various allocations.
%    Entrees under each asset class show the percentage of investors
%    who have most of their IRA
%    and Keogh wealth invested in that particular asset class.
%    The asset class Diversified
%    includes stocks,
%    bonds,
%    and money market investments.
%    The asset class Others
%    include investments in various life insurance products,
%    annuities,
%    real estate, etc.
%    \medskip
%    \footnotesize SOURCE: Survey of Consumer Finances,
%    2001,
%    Federal Reserve Board,
%    USA.\par
%  \end{table}
%}

\begin{table}
    This table contains only text.
    Let's cite Lamport's book here: \cite{Lamport:1994}.
    \caption{%
        This is the caption.
        Let's cite Lamport's book again here: \cite{Lamport:1994}.%
    }
    \label{ta:text-only}
\end{table}

\begin{table}
    % \halign{...} is more flexible than \begin{table}...\end{table}.
    \hbox to \textwidth{%
         \hfill
        \vbox{\halign{
            \strut #&            % 0. \strut
            #\hfil\qquad&        % 1. left
            \hfil #\hfil\qquad&  % 2. center
            \hfil #\cr           % 3. right
            %
            & apple& banana& cherry\cite{Lamport:1994}\cr
            & aardvark& boa constrictor& coyote\cr
        }}
        \hfill
    }
    \caption[short caption for table of contents]{%
        This is a really long and boring caption.
        It goes on and on as if it thinks what it says is important.
        Here is some more of it.
        The citation for ``Lamport::1994'' is ``\cite{Lamport:1994}''.%
    }
    \label{ta:fruit}
\end{table}

% This is loosely based on page 106 of _A Guide to LaTeX_, third edition,
% by Helmut Kopka and Patrick W. Daly.
%\begin{longtable}{|l|l|}
%  \caption{2.2 ``State'' Abbreviations}\\
%  \hline
%  ``State''& Abbreviation\\
%  \hline \endfirsthead
%  \caption[]{\emph{continued}}\\
%  \hline
%  ``State''& Abbreviation\\
%  \hline \endhead
%  \hline
%  \multicolumn{2}{r}{\emph{continued on next page}}
%  \endfoot
%  \hline\endlastfoot
%  Alabama& AL\\
%  Alaska& AK\\
%  American Samoa& AS\\
%  Arizona& AZ\\
%  Arkansas& AR\\
%  Armed Forces Europe& AE\\
%  Armed Forces Pacific& AP\\
%  Armed Forces the Americas& AA\\
%  California& CA\\
%  Colorado& CO\\
%  Connecticut& CT\\
%  Delaware& DE\\
%  District of Columbia& DC\\
%  Federated States of Micronesia& FM\\
%  Florida& FL\\
%  Georgia& GA\\
%  Guam& GU\\
%  Hawaii& HI\\
%  Idaho& ID\\
%  Illinois& IL\\
%  Indiana& IN\\
%  Iowa& IA\\
%  Kansas& KS\\
%  Kentucky& KY\\
%  Louisiana& LA\\
%  Maine& ME\\
%  Marshall Islands& MH\\
%  Maryland& MD\\
%  Massachusetts& MA\\
%  Michigan& MI\\
%  Mississippi& MS\\
%  Missouri& MO\\
%  Montana& MT\\
%  N Minnesota
%  Nebraska& NE\\
%  Nevada& NV\\
%  New Hampshire& NH\\
%  New Jersey& NJ\\
%  New Mexico& NM\\
%  New York& NY\\
%  North Carolina& NC\\
%  North Dakota& ND\\
%  Northern Mariana Islands& MP\\
%  Ohio& OH\\
%  Oklahoma& OK\\
%  Oregon& OR\\
%  Pennsylvania& PA\\
%  Puerto Rico& PR\\
%  Rhode Island& RI\\
%  South Carolina& SC\\
%  South Dakota& SD\\
%  Tennessee& TN\\
%  Texas& TX\\
%  Utah& UT\\
%  Vermont& VT\\
%  Virgin Islands, U.S.& VI\\
%  Virginia& VA\\
%  Washington& WA\\
%  West Virginia& WV\\
%  Wisconsin& WI\\
%  Wyoming& WY\\
%\end{longtable}

\newcommand{\cbackslash}{\char'134}
\newcommand{\copencurly}{\char'173}
\newcommand{\cclosecurly}{\char'175}

\newlength{\twidth}
\newlength{\theight}

\setlength{\twidth}{\textwidth}
\setlength{\theight}{\textheight}

\begin{sidewaystable}
    % The following two lines compensate for what I think is a bug.
    \setlength{\textwidth}{\theight}
    \setlength{\textheight}{\twidth}
    \caption{%
        2.3 sidewaystable mode
        {\tt\cbackslash begin\copencurly table\cclosecurly\/}%
        \ldots
        {\tt\cbackslash end\copencurly table\cclosecurly\/} table%
    }
    \hbox to \textwidth{
        \hfill
        \begin{tabular}{lcr}
            apple& banana& cherry\\
            aardvark& boa constrictor& coyote\\
        \end{tabular}
        \hfill
    }
\end{sidewaystable}

\begin{sidewaystable}
    % The following two lines compensate for what I think is a bug.
    \setlength{\textwidth}{\theight}
    \setlength{\textheight}{\twidth}
    \caption{%
        2.4 sidewaystable mode
        {\tt\cbackslash halign\copencurly ...\cclosecurly\/} table%
    }
    \hbox to \textwidth{%
        \hfill
        \vbox{\halign{
            \strut #&            % 0. \strut
            #\hfil\qquad&        % 1. left
            \hfil #\hfil\qquad&  % 2. center
            \hfil #\cr           % 3. right
            %
            & apple& banana& cherry\cr
            & aardvark& boa constrictor& coyote\cr
        }}
        \hfill
    }
\end{sidewaystable}

\begin{table}
    \begin{tabular}{lcr}
        apple& banana& cherry\\
        aardvark& boa constrictor& coyote\\
        apple& banana& cherry\\
        aardvark& boa constrictor& coyote\\
        apple& banana& cherry\\
        aardvark& boa constrictor& coyote\\
        apple& banana& cherry\\
        aardvark& boa constrictor& coyote\\
        apple& banana& cherry\\
        aardvark& boa constrictor& coyote\\
        apple& banana& cherry\\
        aardvark& boa constrictor& coyote\\
        apple& banana& cherry\\
        aardvark& boa constrictor& coyote\\
        apple& banana& cherry\\
        aardvark& boa constrictor& coyote\\
    \end{tabular}
    \caption{2.5 left hand table}
\end{table}

\begin{table}
    \begin{tabular}{lcr}
        apple& banana& cherry\\
        aardvark& boa constrictor& coyote\\
    \end{tabular}
    \caption{2.6 left hand table}
\end{table}

\begin{sidewaystable}
    % The following two lines compensate for what I think is a bug.
    \setlength{\textwidth}{\theight}
    \setlength{\textheight}{\twidth}
    \caption{%
        2.7 sidewaystable mode
        {\tt\cbackslash begin\copencurly table\cclosecurly\/}%
        \ldots
        {\tt\cbackslash end\copencurly table\cclosecurly\/} table%
    }
    \hbox to \textwidth{%
        \hfill
        \begin{tabular}{lcr}
            apple& banana& cherry\\
            aardvark& boa constrictor& coyote\\
        \end{tabular}
        \hfill
    }
\end{sidewaystable}

%\newlength{\ta}
%\settowidth{\ta}{\vbox{\hbox{Money}\hbox{Market}}}
%\newlength{\tb}
%\settowidth{\tb}{\vbox{\hbox{Stocks}\hbox{and}\hbox{Bonds}}}
%\newlength{\tc}
%\settowidth{\tc}{\vbox{\hbox{Money}\hbox{Market}\hbox{and}\hbox{Stocks}}}
%
%  {\renewcommand{\baselinestretch}{1}
%\begin{table}
%  \caption{\hfil Allocation of the IRA and Keogh Wealth\hfil\break\mbox{}\hfil for Investors With or Without Brokerage Accounts\hfil}
%  \label{tab:ira}
%  \begin{center}
%    \begin{tabular}%
%      {%
%        |%
%        c%
%        |%
%        >{\centering\hspace{0pt}}m{\the\ta}%  Money Market
%        |%
%        c%                                    Stocks 
%        |%
%        c%                                    Bonds
%        |%
%        c%                                    Diversified
%        |%
%        >{\centering\hspace{0pt}}m{\the\tb}%  Stocks and Bonds
%        |%
%        >{\centering\hspace{0pt}}m{\the\tc}%  Money Market and Stocks
%        |%
%        c%                                    Others
%        |%
%      }
%      \hline
%      IMP&
%        Money Market&
%        Stocks&
%        Bonds&
%        Diversified&
%        Stocks and Bonds&
%        Money Market and Stocks&
%        Others\tabularnewline
%      \hline
%      1& 14.19\%& 57.71\%& 12.21\%& 4.50\%& 7.36\%& 3.04\%& 0.99\%\tabularnewline \hline
%      2& 14.08\%& 58.18\%& 12.32\%& 4.44\%& 7.30\%& 2.80\%& 0.88\%\tabularnewline \hline
%      3 &14.26\%& 58.09\%& 12.27\%& 4.50\%& 7.19\%& 2.75\%& 0.94\%\tabularnewline \hline
%      4 &13.94\%& 58.11\%& 12.14\%& 4.78\%& 7.35\%& 2.68\%& 0.99\%\tabularnewline \hline
%      5 &13.92\%& 58.13\%& 11.93\%& 4.56\%& 7.60\%& 2.98\%& 0.88\%\tabularnewline \hline
%    \end{tabular}
%  \end{center}
%  This table presents the allocations of the wealth in the IRA
%  and Keogh accounts in various asset classes.
%  Results from each set of imputed data are presented here.
%  The first column lists the number of the imputations,
%  and rest of the columns lists various allocations.
%  Entrees under each asset class show the percentage of investors
%  who have most of their IRA
%  and Keogh wealth invested in that particular asset class.
%  The asset class Diversified
%  includes stocks,
%  bonds,
%  and money market investments.
%  The asset class Others
%  include investments in various life insurance products,
%  annuities,
%  real estate, etc.
%  \medskip
%  \footnotesize SOURCE: Survey of Consumer Finances,
%  2001,
%  Federal Reserve Board,
%  USA.\par
%\end{table}
%  }

\begin{table}
    \caption{Presidents}
    \begin{center}
        \begin{tabular}{cl}
            \#& Name\\
            1&  George Washington\\
            2&  John Adams\\
            3&  Thomas Jefferson\\
        \end{tabular}
    \end{center}
\end{table}

\begin{table}
    \caption{Presidents with horizontal and vertical lines}
    \begin{center}
        \begin{tabular}{|c|l|}
            \hline
            \#& Name\\
            \hline
            1&  George Washington\\
            \hline
            2&  John Adams\\
            \hline
            3&  Thomas Jefferson\\
            \hline
        \end{tabular}
    \end{center}
\end{table}

%
%  demo-text.tex  2007-07-17  Mark Senn  http://engineering.purdue.edu/~mark
%

\chapter{Demonstrate Text}

% Use single spacing.
\Baselinestretch{1}

% You don't normally need this.
\mbox{}


%\vbox{
\begin{verbatim}
This is a sentence.
This is a sentence.
This is a sentence.
This is a sentence.
This is a sentence.

This is a sentence.
This is a sentence.
This is a sentence.
This is a sentence.
This is a sentence.
\end{verbatim}
This is a sentence.
This is a sentence.
This is a sentence.
This is a sentence.
This is a sentence.

This is a sentence.
This is a sentence.
This is a sentence.
This is a sentence.
This is a sentence.
\vskip\baselineskip
\hrule
%}
\vskip0.5\baselineskip
\filbreak

%\vbox{
\begin{verbatim}
From \verb+http://www.biblegateway.com/passage/?book_id=1&chapter=1&version=50+:

\begin{quote}
    1 In the beginning God created the heavens and the earth.
    2 The earth was without form,
    and void;
    and darkness was on the face of the deep.
    And the Spirit of God was hovering over the face of the waters.

    3 Then God said,``Let there be light'';
    and there was light.
    4 And God saw the light,
    that it was good;
    and God divided the light from the darkness.
    5 God called the light Day,
    and the darkness He called Night.
    So the evening and the morning were the first day. 
\end{quote}
\end{verbatim}
From \verb+http://www.biblegateway.com/passage/?book_id=1&chapter=1&version=50+:

\begin{quote}
    1 In the beginning God created the heavens and the earth.
    2 The earth was without form,
    and void;
    and darkness was on the face of the deep.
    And the Spirit of God was hovering over the face of the waters.

    3 Then God said,``Let there be light'';
    and there was light.
    4 And God saw the light,
    that it was good;
    and God divided the light from the darkness.
    5 God called the light Day,
    and the darkness He called Night.
    So the evening and the morning were the first day. 
\end{quote}
\vskip\baselineskip
\hrule
%}
\vskip0.5\baselineskip
\filbreak

%\vbox{
\begin{verbatim}
\begin{description}
    \item[apple]
        A red fruit.
    \item[banana]
        A yellow fruit.
        This sentence is to make the entry longer so you can see what happens.
        This sentence is to make the entry longer so you can see what happens.
    \item[cherry]
        A red friut.
\end{description}
\end{verbatim}
\begin{description}
    \item[apple]
        A red fruit.
    \item[banana]
        A yellow fruit.
        This sentence is to make the entry longer so you can see what happens.
        This sentence is to make the entry longer so you can see what happens.
    \item[cherry]
        A red friut.
\end{description}
\vskip\baselineskip
\hrule
%}
\vskip0.5\baselineskip
\filbreak

%\vbox{
\begin{verbatim}
\begin{enumerate}
    \item apple
    \item banana
        This sentence is to make the entry longer so you can see what happens.
        This sentence is to make the entry longer so you can see what happens.
    \item cherry
\end{enumerate}
\end{verbatim}
\begin{enumerate}
    \item apple
    \item banana
        This sentence is to make the entry longer so you can see what happens.
        This sentence is to make the entry longer so you can see what happens.
    \item cherry
\end{enumerate}
\vskip\baselineskip
\hrule
%}
\vskip0.5\baselineskip
\filbreak


%\vbox{
\begin{verbatim}
\begin{itemize}
    \item apple
    \item banana
        This sentence is to make the entry longer so you can see what happens.
        This sentence is to make the entry longer so you can see what happens.
    \item cherry
\end{itemize}
\end{verbatim}
\begin{itemize}
    \item apple
    \item banana
        This sentence is to make the entry longer so you can see what happens.
        This sentence is to make the entry longer so you can see what happens.
    \item cherry
\end{itemize}
\vskip\baselineskip
\hrule
%}
\vskip0.5\baselineskip
\filbreak


% Bibliography is required if you consulted any outside references.
% Reference: TM 32.
%
%  bibliography.tex     June 3, 2002     Mark Senn
%
%  This is the bibliography for a simple, example thesis.
%

\bibliography{all}


% Notes and footnotes are optional.
% Reference: TM 34.
% I have not implemented this yet.  Mark Senn 2002-06-03
%%\include{notes}

% A vita is optional for masters theses
% and required for doctoral dissertations.
% Reference: TM 13.
% CHANGE NEXT LINE?
%
%  vita.tex   2003.07.23  14:59:33   Mark Senn <mds@purdue.edu>
%
%  This is the vita for a simple, example thesis.
%
%  A vita is required only in a doctoral dissertation.
%

\begin{vita}
    [Put a brief autobiographical sketch here.]
\end{vita}


\end{document}

% LaTeX won't read after the \end{document} command.
% You can put notes to yourself or LaTeX input not
% ready for use here if you'd like.
