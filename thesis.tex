 %
%  thesis.tex  2011-07-01  Mark Senn  http://engineering.purdue.edu/~mark
%
%  This is the thesis ``root file''.
%
%  To print the final copy of your thesis put a '%'
%  in front of the \includeonly command and type
%  (from page 71 of _LaTeX User's Guide and Reference Manual_, 2nd edition):
%      latex thesis
%      bibtex thesis
%      latex thesis
%      latex thesis
%
%  In "Reference:" listings below:
%      KEY  MEANING
%      TM   ``A Manual for the Preparation of Graduate Theses'',
%             seventh revised edition, The Graduate School, 2006.
%             http://www2.itap.purdue.edu/gradschool//Publications/graduate-thesis-manual.pdf
%      PU   ``A Manual for the Preparation of Graduate Theses'',
%           The Graduate School, Purdue University, 1996.
%           http://www2.itap.purdue.edu/gradschool//Publications/graduate-thesis-manual.pdf
%
%  Search for "CHANGE" below and change things as necessary.
%  I recommend putting "%%" before any existing lines that
%  need to be changed and adding your new line(s) immediately
%  below the existing lines.
%

% See http://www.ecn.purdue.edu/~mark/puthesis/#Options
% for documentclass options.
% CHANGE NEXT LINE?
\documentclass[cs,thesis]{puthesis}

% Define "align" environment used in demo-mathematics.tex.
% CHANGE NEXT LINE?
\usepackage{amsmath}

% Define "multicols" environment environment used in demo-multicols.tex.
% CHANGE NEXT LINE?
\usepackage{multicol}

% Define "subfigure" environment used in "demo-figure.tex".
% CHANGE NEXT LINE?
\usepackage{subfigure}

% Title of thesis (used on cover and in abstract).
% The title shown must be the full, official title of the
% thesis.  Superscripts and subscripts are not permitted in
% the title.
% Reference: TM 26.
% Use \title{Put Title Here} for a one-line title.
% Use \\ to separate lines.
% Put % at the end of the last line to avoid getting an extra space
% in the abstract.
% There are two forms of title: one line or more than one line.
% There are examples of both below.
% Only use one \title.
% CHANGE NEXT FOUR LINES.
\title{Secure Physical System Design Leveraging PUF Technology}

% First author name with first name first is used for cover.
% Second author name with last name first is used for abstract.
% Your full name as it appears in the University records appears
% on the cover.
% Reference: TM 26, 29.
% There are two forms of author, with and without initials.
% There are examples of both below.
% Only use one \author line.
% CHANGE NEXT TWO LINES.
\author{Samuel Kerr}{Kerr, Samuel}

% First is long title of degree (used on cover).
% Second is abbreviation for degree (used in abstract).
% Third is the month the degree was (will be) awarded (used on cover
% and abstract).
% Last is the year the degree was (wlll be) awarded (used on cover
% and abstract).
% The degree title for all doctoral candidates is ``Doctor of Philosophy.''
% The precise degree names for master's candidates appear in the list of
% ``Degrees Offered'' in the Graduate School bulletin.
% The date is the month and year that the degree is actually awarded.
% (If you have registered for ``degree only,'' revise the thesis title
% page to reflect the new date on which the degree is to be awarded.)
% Reference: TM 26--27, 30.
% CHANGE NEXT LINE?
\pudegree{Master of Science}{M.S.}{May}{2012}

% Major professor (used in abstract).
% Use, for example:
%     \majorprof{John Q. Professor}
%     \majorprofs{John Q. Professor and Thomas R. Jones}
%     \majorprofs{John Q. Professor, Thomas R. Jones, and David S. Smith}
% depending on the number of major professors you have.
% CHANGE NEXT LINE.
\majorprof{Elisa Bertino}

% Campus (used only on cover)
% Use one of the following:
%     Fort Wayne
%     Hammond
%     Indianapolis
%     West Lafayette
%     Westville
% Reference: TM 27.
% CHANGE NEXT LINE?
\campus{West Lafayette}

% My command definitions not specific to my thesis.
% CHANGE NEXT LINE?
\input{mydefs}


% My command definitions specific to my thesis.

% CHANGE NEXT LINE TWO LINES?
% Set things up so \margins will show where the margins on the page are.
\newcommand{\margins}{\Repeat{Show where the margins for the page are.}{4}}

% CHANGE NEXT TWO LINES?
% Let typing "\en" be exactly the same as typing "\ensuremath". 
\let\en=\ensuremath

% CHANGE NEXT FIVE LINES?
% Define a \ve command with two arguments, so if it called with
%     \ve an
% it will expand to
%     {\en{a_1},~\en{a_2},\ \ldots,~\en{a_{n}}}
\newcommand{\ve}[2]{\en{#1_1},~\en{#1_2},\ \ldots,~\en{#1_{#2}}}


% To LaTeX only some parts of your thesis put the
% names of the parts to include here.  For example,
% \includeonly{front} would only process front.tex.
% \includeonly{front,introduction} would only process
% front.tex and introduction.tex.
% To print the final copy of your thesis put a '%'
% in front of the \includeonly command and run LaTeX
% three times to make sure that all cross-references
% are correct.  Then run BibTeX once and LaTeX twice
% more.
% CHANGE NEXT LINE?
%\includeonly{front,introduction}

\begin{document}

% Start a new volume for your thesis.  All theses must have at least one
% volume.  If your thesis is too long to fit in one binder put another
% "\volume" between chapters below.
\volume

% Front matter (dedication, etc.).
%
%  revised  front.tex  2011-09-02  Mark Senn  http://engineering.purdue.edu/~mark
%  created  front.tex  2003-06-02  Mark Senn  http://engineering.purdue.edu/~mark
%
%  This is ``front matter'' for the thesis.
%
%  Regarding ``References'' below:
%      KEY    MEANING
%      PU     ``A Manual for the Preparation of Graduate Theses'',
%             The Graduate School, Purdue University, 1996.
%      TCMOS  The Chicago Manual of Style, Edition 14.
%      WNNCD  Webster's Ninth New Collegiate Dictionary.
%
%  Lines marked with "%%" may need to be changed.
%


  % Dedication page is optional.
  % A name and often a message in tribute to a person or cause.
  % References: PU 15, WNNCD 332.
%\begin{dedication}
% This is the dedication.
%\end{dedication}

  % Acknowledgements page is optional but most theses include
  % a brief statement of apreciation or recognition of special
  % assistance.
  % Reference: PU 16.
\begin{acknowledgments}
A lot of people have had an impact in my life, both academic and personal. I wish to thank all of you 
that have influenced it in a positive way.

I would like to especially thank Professor Elisa Bertino. She took a risk on me as a freshmen and allowed 
me to get into research. I have greatly enjoyed learning and growing under her direction these past few 
years. Without such a great mentor, I don't think I would be where I am today.

Michael Kirkpatrick also deserves a lot of thanks. He helped guide some of my initial work with PUF devices,
pointing out things I missed and helping me mature as a researcher. His lessons were invaluable, not
only about the research topic themselves, but also about how research is done.

Finally, my parents deserve a special thanks. Without the wonderful upbringing and strong sense of values 
they instilled in me, none of this would have been possible. I love you Mom and Dad, thanks for pushing
me.

\end{acknowledgments}

  % The preface is optional.
  % References: PU 16, TCMOS 1.49, WNNCD 927.
% \begin{preface}
%  This is the preface.
% \end{preface}

  % The Table of Contents is required.
  % The Table of Contents will be automatically created for you
  % using information you supply in
  %     \chapter
  %     \section
  %     \subsection
  %     \subsubsection
  % commands.
  % Reference: PU 16.
\tableofcontents

  % If your thesis has tables, a list of tables is required.
  % The List of Tables will be automatically created for you using
  % information you supply in
  %     \begin{table} ... \end{table}
  % environments.
  % Reference: PU 16.
\listoftables

  % If your thesis has figures, a list of figures is required.
  % The List of Figures will be automatically created for you using
  % information you supply in
  %     \begin{figure} ... \end{figure}
  % environments.
  % Reference: PU 16.
\listoffigures

  % List of Symbols is optional.
  % Reference: PU 17.
%\begin{symbols}
%  $m$& mass\cr
%  $v$& velocity\cr
%\end{symbols}

  % List of Abbreviations is optional.
  % Reference: PU 17.
\begin{abbreviations}
  PUF& Physically Unclonable Function\cr
  PEAR& Physically Enhanced Authentication Ring\cr
  ROK& Read Once Key\cr
  RS& Reed Solomon Codes\cr
  ZKPK& Zero Knowledge Proof of Knowledge
\end{abbreviations}

  % Nomenclature is optional.
  % Reference: PU 17.
%\begin{nomenclature}
%  Alanine& 2-Aminopropanoic acid\cr
%\end{nomenclature}

  % Glossary is optional
  % Reference: PU 17.
%\begin{glossary}
%	Physical System& A system that interacts with the physical world or has some sort of hardware component to it (e.g. not a pure software implementation)\cr
%  chick& female, usually young\cr
%  dude& male, usually young\cr
%\end{glossary}

  % Abstract is required.
  % Note that the information for the first paragraph of the output
  % doesn't need to be input here...it is put in automatically from
  % information you supplied earlier using \title, \author, \degree,
  % and \majorprof.
  % Reference: PU 17.
\begin{abstract}
Physical systems are systems that interact with the physical world around them. They are becoming increasingly
computationally powerful as faster microprocessors are installed. This allows for many more types of applications
and functionality to be implemented. However, this also means that there is a greater security risk. Much of this
risk has to do with confirming the device as an authentic device or not. This goal is achievable using a technology
known as Physically Unclonable Functions (PUFs). PUFs use the intrinsic differences in hardware behavior to produce a 
random function that is unique to that hardware instance. When combined with existing cryptographic techniques,
these PUFs enable many different types of applications, such
as read once keys, secure communications, and securing smart grids.
\end{abstract}


% Put chapter \include commands here.
% CHANGE \include{...} COMMANDS BELOW?
%
%  revised  introduction.tex  2011-09-02  Mark Senn  http://engineering.purdue.edu/~mark
%  created  introduction.tex  2002-06-03  Mark Senn  http://engineering.purdue.edu/~mark
%
%  This is the introduction chapter for a simple, example thesis.
%

\chapter{Introduction}
\label{chapter:intro}

% Discuss the structure of the thesis
The rest of the thesis is structured as followed. Chapter ~\ref{chapter:physicalsystems} describes physical systems and
their nature, including several of the difficulties that are involved with them. Chapter ~\ref{chapter:cryptographyoverview}
provides some of the necessary cryptography background needed. Chapter ~\ref{chapter:pufoverview} introduces
PUF technology and creates an inital connection to physical systems. Chapter ~\ref{chapter:applications} describes several
applications of PUF technology as it incorporated into physical systems. These applications each demonstrate the use of PUF
as a way of resolving the issues facing physical systems. Chapter ~\ref{chapter:conclusion} draws final conclusions and presents
closing thoughts.


% Dicuss the nature of physical systems
% Discuss the nature of physical system and some of the unique problems associated with them

\chapter{Physical Systems}
\label{chapter:physicalsystems}

Physical systems present an interesting problem domain for study. In contrast to software systems,
they are subjected to multiple different factors that all require consideration during design. Physical
systems frequently must be able to cope with environmental factors such as temperature change, moisture,
or questionable power systems. 

A purely software system may be able to assume it will only receive input from a standard input and output
channel. In contrast, a physical system must be able
to account for multiple different input sources, especially input types that might not have been intended. 
A physical system might receive input directly from end users, networking devices, sensors. A physical
system could also consider environment changes as a sort of secondary, unintended input. For example,
the device's power may fluctuate, potentially changing the behavior of internal circuits. A temperature change
could cause the sensitivity of a certain component to increase or decrease, which will in turn alter the behavior
of the system. These are but a few examples of the various factors that a physical system must be properly
designed to handle and account for.

% Describe some of the typical use cases of physical systems
\section{Typical Organization and Use Case of Physical Systems}
Physical systems are a very broad category that covers a large set of devices, applications, and use cases.
Because of this, it is difficult to discuss physical systems generically. Rather, this section details some
of the common configurations that physical systems take. The rest of the work will regard physical systems as
belonging to one of the configurations discussed.

Despite it being difficult to characterize physical systems in general terms, their operation can be viewed using the mathematical
relation below. Physical inputs are inputs from physical interfaces, such as buttons, control terminals, or radio signals. Other
inputs might be information received over the network or from some sort of attached peripheral.

\begin{align*}
Output = System_{Physical}(Physical Inputs, Other Inputs, Environment)
\end{align*}

The three configurations of physical systems that will be considered are that of the standalone, deployed, or peripheral physical
systems. Each is distinct based upon how much interaction it has, not only with the physical world, but also with other physical
systems or remote devices. Peripheral systems have the most interaction with remote parties and other systems while standalone
systems have the least. In terms of the equation above, the three categories vary based on what sort of 'Other Inputs' are passed
to them.

% Standalone device (in the field)
\subsection{Standalone Physical Systems}
One common configuration of a physical system is that of a standalone physical system. This means that the
physical system is not reliant on communicating with another physical system; it is deployed and functions
independently. An example of this could be a garage door opener. There might be a control pad on the side of the
garage which can open or close the garage door. Additionally, there could be an option to open the door remotely
using some sort of radio frequency device.

Standalone physical systems are more straightforward to deal with in a lot of cases. The garage opener example has
a very specific use case, defined inputs (control pad and remote control) and defined outputs (open or close the garage
door). These qualities typically do not change nor are updated often, if ever. As such, it is typically easy to create a sort
of state diagram to model the behavior of standalone physical systems.

% Device in the field that communicates back
\subsection{Deployed Physical Systems}
Another category of physical systems is that of the 'deployed physical system'. This is a type of physical system that not only
interacts with the environment it is in, but may also communicate with another physical system or some sort of remote
server. A cash register is a good example of a deployed physical system. It takes input from cashiers, who can record transactions,
print receipts, and insert or remove currency from it. However, it also communicates with remote servers in certain cases, such
as when a credit card is used. It must interact with the physical environment, but also must interact with remote servers to verify
the credit card transactions.

Because a deployed physical system must potentially interact with a remote party, it is more complex than a standalone physical
system. It must contend with the same sorts of issues that standalone systems do, but also has to deal with issues that could
relate to the remote communication or other physical system.  As such, it is more complex and difficult to model a deployed
physical system than a standalone physical system.

% Physical peripheral
\subsection{Peripheral Physical Systems}
Peripheral physical systems are the most complex type of physical system. These are normally called 'peripherals'. That
is, they do not provide the main functionality of a system, but augment it's ability in some way. An example could be a 
programmable sensor array. The sensor array could be connected to a network through which it receives commands. The array
would then take sensor readings and communicate them back over the network. Not only does the array have to 
interact with the physical environment to take readings, but there is also the component of dealing with the
command and control element from the network connection.

Peripheral physical systems are characterized by the fact that they not only require interaction with the physical world, but
also with other physical systems or with a remote connection. Because of this, it is very difficult to model the system, since
there are a very large number of ways that the other communicating party could potentially behave, in addition to any difficulties
involved with modeling the physical inputs themselves. 

% Describe the coming sections of the paper

\section{Benign Considerations}
There are multiple ways that a physical system or device could fail. There are multiple benign ways that a system could
fail. That is, the system is not attacked in any way, but some circumstances cause the system to fail or degrade in some
way. There any many different ways to reduce these risks, as discussed below.

% Discuss the benign difficulties associated with physical systems

\subsection{Device Failure}
% Device failure
One failure model is for a complete device failure. In this instance, the device has failed to such a point that it is no
longer able to perform any of its intended function. This typically occurs to some catastrophic component failure or a 
lack of preventative maintenance as a system's performance degrades over time.

To mitigate the danger of a complete system failure, it is necessary to impose a schedule for periodic maintenance and
monitoring of the operating environment for dangerous conditions. An example of this would be inspecting all the moving
parts and springs on a garage door opener to ensure they are not cracking or otherwise at risk of failing. 
Monitoring the environment is critical to ensure that a system is not operating in conditions it was not designed for. If a sensor
array was designed for operating indoors and it is placed outside and subjected to weather, of course it will fail.


\subsection{Device Degradation}
% Device degradation 
One of the few "good" aspects about a complete device failure is that it is readily noticeable. If the system fails completely,
it is not possible to interact with it any more. In contrast, if a device \textit{degrades}, the degradation may not be noticed
for a long time, while in the interim, the degraded system will be used under the assumption it is properly functioning.

An example of this is if a sensor array were to be degraded in some way, its readings might be skewed. The skewed readings
would then be recorded and fed into a processing program or used by some other party. Depending on the application, this
could cause the intended application to then function improperly. In certain instances, this degradation can even prove to be
life threatening, such as when temperature sensors in the Fukishima nuclear plant were incorrectly reporting the internal
temperature of nuclear reactors.~\cite{fukushima}

\section{Malicious Considerations}
% Discuss different malicious problems with physical systems
In addition to the problems that are inherently present in physical systems, it is necessary to consider problems that may
occur from attackers maliciously using the system. They may be attempting to gain unauthorized access to the system,
prevent legitimate users from using the system, disable the system entirely, or any number of other motivations.

\subsection{Denial of Services}
% DOS
It is entirely possible that a malicious entity wants to simply disable and disrupt access to a physical system, preventing legitimate access
to the system. If the physical system is unable to deliver valid services to its intended users, it is essentially worthless. 

Denial of service attacks against physical systems are unique from the denial of service attacks against software. Some are very complex,
while others are very simple. In the simplest case, an attacker can simply use a hammer to smash the system. More complex denial of
service attacks may include inputting erroneous data, which may crash or slow the system. Attackers might also disrupt the environment
that the physical system resides in, such that it is not useful. For instance, an attacker might put a heating element near a thermometer,
which would essentially mean the system is unusable for its intended purpose.

As far as defenses against these types of attacks go, a first step is usually to ensure that the system is protected against a reasonable
amount of tampering. This could include protective cases, placing the system behind a fence, or having a guard present. As mentioned
previously, proper maintenance can also be helpful, to prevent an attacker from manipulating and disturbing the surrounding
environment.

\subsection{Man in the Middle}
% MITM
There is a class of attack known as Man in the Middle (MITM) attacks. This is when an attacker sits between two parties and eavesdrops
on their communications. He is then able to learn sensitive data that the two parties are transmitting.

This type of attack is especially relevant for physical systems. Physical systems frequently transmit data over cables, infrared, radio, or
other wireless communication methods. If an attacker was able to splice a listening device into a cable or construct the appropriate type
of receiver, it is plausible he would be able to easily recover the communications between the two parties. 

Depending on the type of data
being sent, this could compromise the security and integrity of the system. For instance, maybe an attacker would be able to recover the
command sequence to reset a sensor array. He could then reset the sensor array at will.

To combat this type of threat, it is important to assume that any communication being done is being eavesdropped on. This then
necessitates \textit{encrypting} the data being transmitted. In this way, even if an adversary was to recover the data, he would not be
able to make sense of it. Chapter~\ref{chapter:cryptographyoverview} goes into more detail on encryption techniques.

As a final note, it is important to note that MITM are not relevant for only signals being transmitted over wireless and wires, but also on
the internal buses of the circuits themselves. An attacker might be able to attach logic probes to bus lines between the processor and memory
of the system and deduce sensitive data. In this case, it is important to take measures to prevent these bus lines from being exposed,
through the use of potting and other tamper-proofing methods. Another option is to incorporate the entire design (or at least the sensitive
bus lines) on a single chip, such as a Field Programmable Gate Array (FPGA) or a System on a Chip (SOC).

\subsection{Impersonation}
% Impersonation
Another issue for physical system designers to be aware of is that any party they are communicating with is actually an authorized party.
This is especially relevant for deployed physical systems and peripheral physical systems. Since they require external communication as a
major component of their proper operation, they are especially sensitive to these attacks.

It is plausible that an attacker could disconnect the cables used to communicate and re-attach them to his own machine. He could then
issue commands and communicate with the physical system. Unless protective measures are in place, the system would then interact
back with the attacker. The attacker could then issue any sorts of commands that he wished of the system.

The issue of impersonation harkens to the need for \textit{authentication}. Chapter~\ref{chapter:cryptographyoverview} goes into more
details on authentication protocols, but essentially, all communications between the system and the other party would have to be
\textit{signed}. If the signatures did not match the expected values, the communication is rejected and dropped. In this way, an attacker
would have to be able to forge the signature of the valid party, which is considered computationally difficult if a proper signature scheme
is used.

% Replay attacks
\subsection{Replay Attacks}
There is a class of attacks that is related to MITM attacks called replay attacks. Replay attacks leverage the fact that certain protocols
might consistently send the same data every execution of the protocol. For instance, consider a system that requires the sender to 
send an encrypted version of an ID number
before every message to identify itself. If that ID is always the same, an attacker could simply capture the encrypted text and send
that; he does not need to actually know the plaintext version of the ID to impersonate the sending party.

This type of attack can be remedied by ensuring that every execution of a protocol is unique. This is done through the use of
time stamps or "nonce" values, which are randomly chosen, one time use values. Then, if an attacker tried to replay previous
communications, the attack would fail since the time stamp or the nonce value would not match. So in the example above, the
sender might encrypt his ID number concatenated with the current time. Then, an attacker would not be able to re-use any
communications he captures in the future.

% Signals
\subsection{Signals Injection}
Due to the nature of electronics, physical systems are susceptible to external signals being directed at them. If an electric or 
magnetic field is directed at certain elements of internal circuitry, it is possible to alter the behavior of those circuits. An attacker
can potentially bombard a physical system in some way to elicit a response from the device.

An example of this type of signal injection was shown in the Cold War with 'The Thing'.~\cite{thing} In 1945, a Soviet made Seal of the 
Republic was given as a gesture of friendship and installed in a sensitive office. When bombarded with radio waves, the device internals
would resonate, modulate the radio waves, and it was possible to listen to conversations in the room where it was installed. This is a
classical example of how physical systems can be manipulated through signal injection. In this case, the signal injection was a desired
feature, but it is important to be aware of this danger when designing physical systems.

Another example is disruption of GPS signals. This typically occurs because radio frequency signals are using the same wavelengths as
GPS signals. GPS signals are usually weaker than RF signals, so the RF signals dominate and drown out the GPS signals. ~\cite{gpsdisruption} 
was delivered in 2001 and details some of these risks and defenses associated with GPS interference, both unintentional and intentional.

To mitigate signal injection, it is important that system designers consider and plan for signal injection attacks. Defenses against this could
include shielding equipment against magnetic and electrical fields or using multiple frequencies and receivers when possible~\cite{gpsdisruption}.
These are just some techniques to defend against the signal injection threat which must be considered.

\subsection{Signal Emissions}
Adversaries may also attempt to harvest a physical systems signal emissions in an attempt to gather information. This is because during
normal operation, many devices give off electromagnetic and radio signals, at least to some extent, even if unintended. There has been
examples where it is possible to recreate what is on a user's CRT or LCD computer monitor by recording the emissions of the monitor
from a far, using a process known as "Van Eck phreaking".~\cite{monitor}~\cite{lcds}

NATO created a program called TEMPEST to investigate and report on the risks associated with signal emissions and defenses
against these threats.~\cite{tempest} Some of the easily implementable changes they suggest are to put electromagnetic shielding
around devices. Suggestions presented also include signal filtering such that certain frequencies are attenuated or completely removed
from emission.

\subsection{Tampering}
If a system is not protected, tampering with the system itself is one of the easiest attacks for an adversary to execute.
He can attach logic probes or some device to record traffic being sent over signal buses to learn sensitive information.
He could also tamper with certain portions of the system so that error handling and recovery routines were triggered,
which might be easier to exploit in some way.

There are many different techniques to deal with physical tampering of a system. One of these includes potting, which 
involves sealing all components in a type of epoxy, so that no wires are exposed. Another technique would be to
put the sensitive components in an enclosure that had some sort of alarm on it. When the enclosure was opened, an
authority would be notified, who could then deal with the tampering. Physical tampering is a very large problem
and these are just a few techniques to address it, but every system designer should consider how to protect his
system from tampering.


% Describe the cryptographic background
\chapter{Cryptography Overview}
\label{chapter:cryptographyoverview}

% Provide a brief overview
\section{Overview}
Before delving into the details of various applications, it is necessary to define and understand several
cryptographic primitives, as they form a foundation to build off. The following sections present a brief
introduction to the necessary cryptographic primitives that will be used in the rest of the thesis.

% Encryption operations
\section{Encryption}
It is often necessary to scramble and protect data so that only certain parties, such as those who
possess a key value, can de-scramble and read the protected data. This might be necessary when
sending any sort of sensitive data, such as financial records or e-mail messages. Presumably, if a 
person does not have the correct key value, he or she will not be able to scramble or unscramble
the data properly.

The act of scrambling the data is called \emph{encryption}. The corresponding act of descrambling
encrypted data is callled \emph{decryption}.

Encryption and decryption operations and relevant parameters are denoted using the following notation below.

\begin{align}
C  = E_K(M) \\
M = D_K(C)
\end{align}

Above, C is the \emph{ciphertext}, or encrypted text.  
M is the message or \emph{plaintext}. 
K represents the \emph{key value}. 
E represents the \emph{encryption algorithm}, of which there are several types. This algorithm takes plaintext as a parameter and returns ciphertext.
D represents the \emph{decryption algorithm}, which takes ciphertext and return plaintext.

A sender would use his plaintext message to generate the ciphertext and transmit it. The receiver would then process the received data using the decryption
algorithm and then be able to successfully recover the plaintext message.

% Symmetric key encryption
\subsection{Symmetric Encryption}
Symmetric encryption is a fairly intuitive method of using encryption. In this style of encryption,
both the sender and receiver share the same key value, K.

% Asymmetric key encryption
\subsection{Asymmetric Encryption}

% ZKPK
\section{Zero Knowledge Proof of Knowledge}

% Committments
\section{Commitment Schemes}


% Describe PUFs at a general level
% Give an overview of PUF devices

\chapter{Physically Unclonable Functions}
\label{chapter:pufoverview}
It is desirable for a user to be sure that the device that he is using is authentic. However, due to the sophistication
of forgeries or possible communication tampering, a user might be suspicious that the system is the system it claims
to be. A device called a Physically Unclonable Function, or PUF, is a technology that remedies this problem.

A PUF device provides a unique challenge-response capability. That is, when two PUFS are provided the identical
challenge, they will each produce unique responses. In this way, a PUF, and the system it contains, 
can be identified by the response value it generates to a specific challenge. A more formalized definition of
this relationship is given below.

\begin{align*}
PUF_1(C) = R_1\\
PUF_2(C) = R_2\\
R_1 \neq R_2
\end{align*}



\section{Types of PUFs}
A PUF device provides this sort of relationship by leveraging the physical properties
of the materials in which it is instantiated. There are several different ways of doing
this, from measuring the distortions of reflected light to leveraging the
manufacturing inconsistencies from one chip to another.

The Ring Oscillator PUF is presented first and in somewhat greater detail than
other types of PUF since this is the type of PUF that the author worked with primarily.
As such, it was incorporated in many of the different applications presented later in
Chapter ~\ref{chapter:applications}.

\subsection{Ring Oscillator PUF}
A Ring Oscillator PUF is a PUF design that utilizes a circuit called a Ring 
Oscillator (RO). An RO is an odd number of inverter gates tied together. Because
there are an odd number of gates, this will produce a continuously changing,
or oscillating, signal. 

\begin{figure}[h] % The h specifies to place the figure 'here' as in, inline with the source code
\includegraphics[]{images/ro.png}
\caption{A 3 gate Ring Oscillator}
\label{fig:ro}
\end{figure}

Depending on the number of inverter gates being used
as well as the propogation delay of every indiviudal inverter, the output
frequency of one RO may be different from another RO. In Figure ~\ref{fig:ro},
this output signal corresponds to the signal marked $Q$.

When used as part of a PUF, the unique behavior of an RO will be examined.
Consider again the 3 stage RO as shown in Figure ~\ref{fig:ro}. All three
inverter gates are assumed to have the same propogation delay and the
interconnecting wires are assumed to impose a neglible delay. However, in
an actual instationtion of an RO, these assumptions are invalid. All three inverters
should have the same propogation delay, but, due to uncontrollable manufacturing
inconsistencies and tolerances, they do not. In a similar vein, the interconnecting
wires will also impose a non-zero delay time in signal propogation. Both of these
factors will combine so that even if two ROs are produced on the same 
manufacturing line, they will generate a slightly different output frequency.

The slightly different output frequenices of two ring oscillators forms the basis
of randomness for the Ring Oscillator PUF. Because the output frequencies of the
ROs cannot be predicted, their actual frequency at runtime gives a way to uniquely
identify the individual PUF that contains them. In Figure ~\ref{fig:ropuf}, a more
detailed diagram of a PUF based off of ring oscillators is presented.

\begin{figure}[h]
\includegraphics[width=500px]{images/ropuf.png}
\caption{A Basic Ring Oscillator Based PUF}
\label{fig:ropuf}
\end{figure}

As can be seen in Figure ~\ref{fig:ropuf}, the output of two ring oscillators is
fed into two multiplexers. Based on the challenge value, the multiplexers select
one of the output values and feed those into a counter. The counters measure the amount
of oscillations that occur over a given time period. At the end of the time period,
if the top counter has more oscillations, a 0 is recorded as the response, otherwise
a 1 is recorded as the response.

While the diagram only displays two ring oscillators and only 1 bit of challenge and
response, this diagram can be extrapolated to form arbitrarily large PUFs. Trivially,
the design can simply be copied so that an N-bit PUF requires 2N ring oscillators,
however, there are alternative methods that require a lesser amount of ring oscillators,
which in turn requires less space on a chip, which is desirable.

It is important to note that this PUF is not concerned with the absolute number of oscillations
of ring oscillators, but rather, with the \emph{relative} number of oscillations between
the two ring oscillators. This is because the counters are compared to each other, rather than
some absolute constant. This provides the benefit that if the entire circuit is heated or cooled,
which will in turn affect operating frequency, the PUF will still produce the same response, since
both oscillators were effected in the same way.

\subsection{Butterfly PUF}

\begin{figure}[h]
\includegraphics[width=500px]{images/butterflypuf.png}
\end{figure}

~\cite{butterflypuf}
\subsection{Optical PUF}

\subsection{Coating PUF}





\section{Vulnerabilities}
% Differential power analysis (security engineering, chapter 15)

% Tampering

\section{Comparison to Alternatives}

\subsection{Trusted Platform Module}

\subsection{Radio Frequency Identification Tags}



% Discuss the different PUF applications and how they relate to physical system design
% Give a general introduction to the PUF applications chapter

\chapter{Applications}
\label{chapter:applications}

~\cite{securityengineering}
% Discuss the PUF ROK and how it is relevant

\section{Read Once Keys}
\label{section:rok} % PUF ROK work
% Discuss the PEAR work and how it is relevant

\chapter{Physically Enhanced Authentication Ring}
\label{chapter:pear}

\section{Overview}
One problem that is present when using computers is that users typically are not aware of the security of the system
they are using. For instance, an attacker could have installed a key logger on a user's system to harvest every username
and password they have. Even with the best security systems on the machine in place, if the attacker is able to capture
a user's keystrokes, the other security is moot. 

PEAR, or Physically Enhanced Authentication Ring, was designed to counteract this key logger threat to a system. In addition
to defending against keyloggers specifically, it increases security in general because it is the second part of a "two factor
authentication" system. It also is a physical system, specifically a peripheral physical system, since it incorporates its
own processing and interacts with the user's normal computer system. Thirdly, the PEAR system incorporates a PUF device,
so it is a good example of when PUF technology is useful.

From a high level perspective, a PEAR device is a device consisting of a PUF, a keypad, and some supporting circuitry. When
a user wishes to log on to a given service, rather than using the keyboard for a password, he enters a 4 digit PIN on the PEAR
device. The PEAR device then executes the PUF and then initiates a zero-knowledge proof of knowledge with the service
provider. Note that no sensitive data is actually input to the PC, which potentially has a keylogger. Any data that the PC
is requested to ferry between the PEAR device and the service provider is encrypted, so recording this data does not reveal
any information.

The system works by having every service provider associated with an ID number of some kind. Each user of the service will
also have an ID number associated with it. This allows both parties to identify themselves to each other.

\section{Protocol Details}
The PEAR system consists of two parts, an enrollment step initially and then an authentication step.
Table~\ref{tab:pearprotocol} presents a formalized description of the protocols, while Figure~\ref{fig:pearauthentication}
and Figure~\ref{fig:pearenrollment} give graphical representations that occur.

An interesting point to note is that during the enrollment stage, an "out of band" communication is required to deliver
the combination of the service provider's ID, the user's corresponding ID for that service, and a nonce value. This could
be done by installing these values on the hardware device before it is given to an end user. For instance, if PEAR was being
used with a bank, the bank might install these values before mailing the device to the user.

\begin{figure}[!ht]
\includegraphics[width=500px]{images/enrollment.jpg}
\label{fig:pearenrollment}
\caption{The enrollment stage of PEAR}
\end{figure}
\FloatBarrier

\begin{figure}[!ht]
\includegraphics[width=500px]{images/auth.jpg}
\label{fig:pearauthentication}
\caption{The authentication stage of PEAR}
\end{figure}
\FloatBarrier

\begin{table}[!ht]
\label{tab:pearprotocol}
\caption{Formalized version of the PEAR protocols}
\noindent\makebox[\textwidth]{%
\begin{tabular}{|l|}
\hline
{\sf Enroll}($U$) - Device $T$ (using input data from user $U$) computes a commitment and enrolls the results with $S$. \\
\hline
- $C$ requests enrollment from $S$ \\
- $S$ sends the tuple $<$Label, ID$>$ and nonce $N$ to $T$ over a secure channel \\
- $U$ sends PIN to $T$ \\
- $T$ computes {\sf H}(ID, Label, PIN) as $H_{result}$ \\
- $T$ executes {\sf PGV}($H_{result}$) as $P_{result}$ \\
- $T$ sends {\sf Commit}($P_{result}$), $<$Label, ID$>$, {\sf H}({\sf Commit}($P_{result}$),Label,ID,$N$) to $S$, via $C$ \\
\hline
\hline
{\sf Authenticate}($U$) - Device $T$ (using input data from user $U$) authenticates itself as a registered user of $S$. \\
\hline
- $C$ initiates the authentication request from $S$ \\
- $S$ sends the tuple $<$Label, ID$>$ and {\sf Chal}($P_{result}$) to $T$ \\
- $U$ sends PIN to $T$ \\
- $T$ computes {\sf H}(ID, Label, PIN) as $H_{result}$ \\
- $T$ executes {\sf PGV}($H_{result}$) as $P_{result}$ \\
- $T$ responds with {\sf Prove}($P_{result}$), which $C$ forwards to $S$ \\
\hline
\end{tabular}
}
\end{table}
\FloatBarrier

\section{Security Considerations}
Several lemmas are presented below which address various different security aspects of the PEAR system.
Following the lemmas is a discussion of some of the different security issues facing physical systems that
were discussed in Chapter~\ref{chapter:physicalsystems}.

\subsection{Lemmas}


\noindent \textbf{Lemma 1.} \\
\noindent \emph{A man-in-the-middle attacker cannot recover any useful data communicated over the network between the
service provider and the computer.} \\
{\bf Proof:}  The only data that is transmitted between the computer and network is the tuple containing the service
ID and the user's ID initially and then steps of the zero-knowledge proof (see Figures~\ref{fig:enroll} and
\ref{fig:auth}). The tuple will only be sent during authentication, so we can assume that users are already enrolled.
An attacker gains nothing by intercepting the tuple during authentication, since it still requires both the user PIN
number and the device itself to impersonate a user. Intercepting the steps of the zero-knowledge proof also gives him
no information since these zero-knowledge protocols do not reveal any information about the committed value.
%\hfill$\square$ \\

\noindent \textbf{Lemma 2.} \\
\noindent \emph{A man-in-the-middle attacker cannot recover any useful data communicated between the user computer and
the device.} \\
{\bf Proof:}  As shown in Figures~\ref{fig:enroll} and \ref{fig:auth}, the only data that is being transmitted over this channel is
the tuple from the server and the zero-knowledge steps. As shown in Lemma 1, an attacker cannot gain any useful
information from this. 
Also note that the PUF secret is never transferred outside of the device, but rather a commitment or
proof is sent. As such, a MITM attack would not reveal the user's secret, but only the various steps of
the zero-knowledge proofs, which are secure against MITM attacks. In addition, the service provider does not even ever
know the user's PIN.%\hfill$\square$ \\

\noindent \textbf{Lemma 3.} \\
\noindent \emph{An active man-in-the-middle attacker cannot recover any useful information by modifying data between
the device and computer or computer and network during the authentication stage.} \\
{\bf Proof:}  An attacker who modifies the tuple being sent to the device or computer from the network would cause the
device to create an incorrect zero-knowledge proof. This would disrupt the user's ability to authenticate. However,
the attacker would not be able to glean any information from the proof generated from this modified tuple, due to the
use of the zero-knowledge proof.
Note that an attacker would be able to recover useful information if it could modify the tuple during enrollment. It
could substitute a malicious tuple for the valid tuple, which would cause users to be authenticating to the MITM, rather
than the service provider. We avoid this problem by requiring that the tuple be sent securely during enrollment.
%\hfill$\square$ \\

\noindent \textbf{Lemma 4.} \\
\noindent \emph{A PPT adversary can impersonate a legitimate user to the server with only negligible probability.} \\
{\bf Proof:}   As the final authentication step is to complete a zero-knowledge proof, an attacker would have to be able
to defeat a zero-knowledge proof, which happens only with a negligible probability if an attacker does not know the
user's secret.%\hfill$\square$ \\

\noindent \textbf{Lemma 5.} \\
\noindent \emph{Given physical access to the device, an attacker could impersonate the legitimate user with only
negligible probability.} \\
{\bf Proof:}  If an attacker had access to the device, it would not be able to compute the proper hash value unless it
supplied the correct PIN to the device. If the attacker attempted a brute force attack on the user's PIN,
it would be trivial for a server to detect and disable the user's account temporarily. As long as the key space for the
PIN is sufficient, this attack is not realistic.%\hfill$\square$ \\

\noindent \textbf{Lemma 6.} \\
\noindent \emph{A legitimate user can authenticate to a legitimate $S$, except with negligible probability.} \\
{\bf Proof:}  As a legitimate user would have access to the user PIN and a valid tuple from the server, he would be
able to successfully complete the zero-knowledge proof, thus authenticating.%\hfill$\square$ \\

\noindent \textbf{Lemma 7.} \\
\noindent \emph{An attacker cannot enroll using an existing or past user's credentials, except with negligible
probability.} \\
{\bf Proof:} An attacker would be able to capture a user's tuple during authentication. It is plausible that he could
attempt to enroll using this tuple. To prevent this, when the service provider issues the tuple initially, it also
provides a nonce. During the enrollment protocol, the user submits the committed value, the tuple, and a hash of the
tuple, nonce, and committed value. The service provider will verify that this tuple is valid. If the tuple is not
valid or has already been enrolled, the service provider denies the enrollment request.%\hfill$\square$ \\

\subsection{Man in the Middle}

\subsection{Replay Attacks}

\subsection{Impersonation}

\section{Implementation}
From a high level view, Figure~\ref{fig:peararchitecture} describes the architecture of a PEAR enabled device.

\begin{figure}[!ht]
\includegraphics[width=500px]{images/pearimpl.jpg}
\label{fig:peararchitecturet}
\caption{Implementation of a PEAR device}
\end{figure}
\FloatBarrier

\section{Acknowledgement}
This work was partially funded by Sypris Electronics. A paper on PEAR was published in 2010 in the SPRINGL  % PEAR work
% Discuss the DOE project and how it is relevant

\chapter{Smart Grid and Smart Meters}
\label{chapter:doe} % DOE work

% Summary and/or conclusions are optional but often used.
% The summary and/or conclusions often are the last
% major division(s) of the text.
% Reference: TM 32.
% CHANGE NEXT LINE?
%
%  summary.tex  2007-02-06  Mark Senn  http://www.ecn.purdue.edu/~mark
%

\chapter{Summary}

This is the summary chapter.


% Recommendations are optional.
% You may include recommendations as a major division if your
% subject matter and research dictate.
% Reference: TM 32.
% CHANGE NEXT LINE?
%\include{recommendations}

% Appendices are optional.
% Appendices are not necessarily part of every thesis. Appendices are used
% for supplementary illustrative material, original data, computer programs,
% and other material not necessarily appropriate for inclusion within the
% text of your thesis. 
% Reference: TM 33.
% Use "\appendix" for one appendix or "\appendices" for more than one
% appendix.
% CHANGE NEXT 7 LINES?
\appendices
%\include{demo-citations}
%\include{demo-figures}
%\include{demo-mathematics}
%\include{demo-multicols}
%%
%  demo-tables.tex  2009-09-29  Mark Senn  http://engineering.purdue.edu/~mark
%
%  Demonstrate how to do tables.
%

\chapter{Demonstrate Tables}

\begin{tabular}{ll}
    \bf Label& \bf Number\\
    ta:text-only& \ref{ta:text-only}\\
    ta:fruit&     \ref{ta:fruit}
\end{tabular}

\newlength{\ta}
\newlength{\tb}
\newlength{\tc}

\settowidth{\ta}{\vbox{\hbox{Money}\hbox{Market}}}
\settowidth{\tb}{\vbox{\hbox{Stocks}\hbox{and}\hbox{Bonds}}}
\settowidth{\tc}{\vbox{\hbox{Money}\hbox{Market}\hbox{and}\hbox{Stocks}}}

%{\renewcommand{\baselinestretch}{1}
%  \begin{table}
%    \caption{%
%      \hfil Allocation of the IRA and Keogh Wealth\hfil\break
%      \mbox{}\hfil for Investors With or Without Brokerage Accounts\hfil
%    }
%    \label{tab:ira}
%    \begin{center}
%      \begin{tabular}%
%        {%
%          |%
%          c%
%          |%
%          >{\centering\hspace{0pt}}m{\the\ta}%  Money Market
%          |%
%          c%                                    Stocks 
%          |%
%          c%                                    Bonds
%          |%
%          c%                                    Diversified
%          |%
%          >{\centering\hspace{0pt}}m{\the\tb}%  Stocks and Bonds
%          |%
%          >{\centering\hspace{0pt}}m{\the\tc}%  Money Market and Stocks
%          |%
%          c%                                    Others
%          |%
%        }
%        \hline
%        IMP&
%          Money Market&
%          Stocks&
%          Bonds&
%          Diversified&
%          Stocks and Bonds&
%          Money Market and Stocks&
%          Others\tabularnewline
%        \hline
%        1& 14.19\%& 57.71\%& 12.21\%& 4.50\%& 7.36\%& 3.04\%& 0.99\%\tabularnewline \hline
%        2& 14.08\%& 58.18\%& 12.32\%& 4.44\%& 7.30\%& 2.80\%& 0.88\%\tabularnewline \hline
%        3 &14.26\%& 58.09\%& 12.27\%& 4.50\%& 7.19\%& 2.75\%& 0.94\%\tabularnewline \hline
%        4 &13.94\%& 58.11\%& 12.14\%& 4.78\%& 7.35\%& 2.68\%& 0.99\%\tabularnewline \hline
%        5 &13.92\%& 58.13\%& 11.93\%& 4.56\%& 7.60\%& 2.98\%& 0.88\%\tabularnewline \hline
%      \end{tabular}
%    \end{center}
%    This table presents the allocations of the wealth in the IRA
%    and Keogh accounts in various asset classes.
%    Results from each set of imputed data are presented here.
%    The first column lists the number of the imputations,
%    and rest of the columns lists various allocations.
%    Entrees under each asset class show the percentage of investors
%    who have most of their IRA
%    and Keogh wealth invested in that particular asset class.
%    The asset class Diversified
%    includes stocks,
%    bonds,
%    and money market investments.
%    The asset class Others
%    include investments in various life insurance products,
%    annuities,
%    real estate, etc.
%    \medskip
%    \footnotesize SOURCE: Survey of Consumer Finances,
%    2001,
%    Federal Reserve Board,
%    USA.\par
%  \end{table}
%}

\begin{table}
    This table contains only text.
    Let's cite Lamport's book here: \cite{Lamport:1994}.
    \caption{%
        This is the caption.
        Let's cite Lamport's book again here: \cite{Lamport:1994}.%
    }
    \label{ta:text-only}
\end{table}

\begin{table}
    % \halign{...} is more flexible than \begin{table}...\end{table}.
    \hbox to \textwidth{%
         \hfill
        \vbox{\halign{
            \strut #&            % 0. \strut
            #\hfil\qquad&        % 1. left
            \hfil #\hfil\qquad&  % 2. center
            \hfil #\cr           % 3. right
            %
            & apple& banana& cherry\cite{Lamport:1994}\cr
            & aardvark& boa constrictor& coyote\cr
        }}
        \hfill
    }
    \caption[short caption for table of contents]{%
        This is a really long and boring caption.
        It goes on and on as if it thinks what it says is important.
        Here is some more of it.
        The citation for ``Lamport::1994'' is ``\cite{Lamport:1994}''.%
    }
    \label{ta:fruit}
\end{table}

% This is loosely based on page 106 of _A Guide to LaTeX_, third edition,
% by Helmut Kopka and Patrick W. Daly.
%\begin{longtable}{|l|l|}
%  \caption{2.2 ``State'' Abbreviations}\\
%  \hline
%  ``State''& Abbreviation\\
%  \hline \endfirsthead
%  \caption[]{\emph{continued}}\\
%  \hline
%  ``State''& Abbreviation\\
%  \hline \endhead
%  \hline
%  \multicolumn{2}{r}{\emph{continued on next page}}
%  \endfoot
%  \hline\endlastfoot
%  Alabama& AL\\
%  Alaska& AK\\
%  American Samoa& AS\\
%  Arizona& AZ\\
%  Arkansas& AR\\
%  Armed Forces Europe& AE\\
%  Armed Forces Pacific& AP\\
%  Armed Forces the Americas& AA\\
%  California& CA\\
%  Colorado& CO\\
%  Connecticut& CT\\
%  Delaware& DE\\
%  District of Columbia& DC\\
%  Federated States of Micronesia& FM\\
%  Florida& FL\\
%  Georgia& GA\\
%  Guam& GU\\
%  Hawaii& HI\\
%  Idaho& ID\\
%  Illinois& IL\\
%  Indiana& IN\\
%  Iowa& IA\\
%  Kansas& KS\\
%  Kentucky& KY\\
%  Louisiana& LA\\
%  Maine& ME\\
%  Marshall Islands& MH\\
%  Maryland& MD\\
%  Massachusetts& MA\\
%  Michigan& MI\\
%  Mississippi& MS\\
%  Missouri& MO\\
%  Montana& MT\\
%  N Minnesota
%  Nebraska& NE\\
%  Nevada& NV\\
%  New Hampshire& NH\\
%  New Jersey& NJ\\
%  New Mexico& NM\\
%  New York& NY\\
%  North Carolina& NC\\
%  North Dakota& ND\\
%  Northern Mariana Islands& MP\\
%  Ohio& OH\\
%  Oklahoma& OK\\
%  Oregon& OR\\
%  Pennsylvania& PA\\
%  Puerto Rico& PR\\
%  Rhode Island& RI\\
%  South Carolina& SC\\
%  South Dakota& SD\\
%  Tennessee& TN\\
%  Texas& TX\\
%  Utah& UT\\
%  Vermont& VT\\
%  Virgin Islands, U.S.& VI\\
%  Virginia& VA\\
%  Washington& WA\\
%  West Virginia& WV\\
%  Wisconsin& WI\\
%  Wyoming& WY\\
%\end{longtable}

\newcommand{\cbackslash}{\char'134}
\newcommand{\copencurly}{\char'173}
\newcommand{\cclosecurly}{\char'175}

\newlength{\twidth}
\newlength{\theight}

\setlength{\twidth}{\textwidth}
\setlength{\theight}{\textheight}

\begin{sidewaystable}
    % The following two lines compensate for what I think is a bug.
    \setlength{\textwidth}{\theight}
    \setlength{\textheight}{\twidth}
    \caption{%
        2.3 sidewaystable mode
        {\tt\cbackslash begin\copencurly table\cclosecurly\/}%
        \ldots
        {\tt\cbackslash end\copencurly table\cclosecurly\/} table%
    }
    \hbox to \textwidth{
        \hfill
        \begin{tabular}{lcr}
            apple& banana& cherry\\
            aardvark& boa constrictor& coyote\\
        \end{tabular}
        \hfill
    }
\end{sidewaystable}

\begin{sidewaystable}
    % The following two lines compensate for what I think is a bug.
    \setlength{\textwidth}{\theight}
    \setlength{\textheight}{\twidth}
    \caption{%
        2.4 sidewaystable mode
        {\tt\cbackslash halign\copencurly ...\cclosecurly\/} table%
    }
    \hbox to \textwidth{%
        \hfill
        \vbox{\halign{
            \strut #&            % 0. \strut
            #\hfil\qquad&        % 1. left
            \hfil #\hfil\qquad&  % 2. center
            \hfil #\cr           % 3. right
            %
            & apple& banana& cherry\cr
            & aardvark& boa constrictor& coyote\cr
        }}
        \hfill
    }
\end{sidewaystable}

\begin{table}
    \begin{tabular}{lcr}
        apple& banana& cherry\\
        aardvark& boa constrictor& coyote\\
        apple& banana& cherry\\
        aardvark& boa constrictor& coyote\\
        apple& banana& cherry\\
        aardvark& boa constrictor& coyote\\
        apple& banana& cherry\\
        aardvark& boa constrictor& coyote\\
        apple& banana& cherry\\
        aardvark& boa constrictor& coyote\\
        apple& banana& cherry\\
        aardvark& boa constrictor& coyote\\
        apple& banana& cherry\\
        aardvark& boa constrictor& coyote\\
        apple& banana& cherry\\
        aardvark& boa constrictor& coyote\\
    \end{tabular}
    \caption{2.5 left hand table}
\end{table}

\begin{table}
    \begin{tabular}{lcr}
        apple& banana& cherry\\
        aardvark& boa constrictor& coyote\\
    \end{tabular}
    \caption{2.6 left hand table}
\end{table}

\begin{sidewaystable}
    % The following two lines compensate for what I think is a bug.
    \setlength{\textwidth}{\theight}
    \setlength{\textheight}{\twidth}
    \caption{%
        2.7 sidewaystable mode
        {\tt\cbackslash begin\copencurly table\cclosecurly\/}%
        \ldots
        {\tt\cbackslash end\copencurly table\cclosecurly\/} table%
    }
    \hbox to \textwidth{%
        \hfill
        \begin{tabular}{lcr}
            apple& banana& cherry\\
            aardvark& boa constrictor& coyote\\
        \end{tabular}
        \hfill
    }
\end{sidewaystable}

%\newlength{\ta}
%\settowidth{\ta}{\vbox{\hbox{Money}\hbox{Market}}}
%\newlength{\tb}
%\settowidth{\tb}{\vbox{\hbox{Stocks}\hbox{and}\hbox{Bonds}}}
%\newlength{\tc}
%\settowidth{\tc}{\vbox{\hbox{Money}\hbox{Market}\hbox{and}\hbox{Stocks}}}
%
%  {\renewcommand{\baselinestretch}{1}
%\begin{table}
%  \caption{\hfil Allocation of the IRA and Keogh Wealth\hfil\break\mbox{}\hfil for Investors With or Without Brokerage Accounts\hfil}
%  \label{tab:ira}
%  \begin{center}
%    \begin{tabular}%
%      {%
%        |%
%        c%
%        |%
%        >{\centering\hspace{0pt}}m{\the\ta}%  Money Market
%        |%
%        c%                                    Stocks 
%        |%
%        c%                                    Bonds
%        |%
%        c%                                    Diversified
%        |%
%        >{\centering\hspace{0pt}}m{\the\tb}%  Stocks and Bonds
%        |%
%        >{\centering\hspace{0pt}}m{\the\tc}%  Money Market and Stocks
%        |%
%        c%                                    Others
%        |%
%      }
%      \hline
%      IMP&
%        Money Market&
%        Stocks&
%        Bonds&
%        Diversified&
%        Stocks and Bonds&
%        Money Market and Stocks&
%        Others\tabularnewline
%      \hline
%      1& 14.19\%& 57.71\%& 12.21\%& 4.50\%& 7.36\%& 3.04\%& 0.99\%\tabularnewline \hline
%      2& 14.08\%& 58.18\%& 12.32\%& 4.44\%& 7.30\%& 2.80\%& 0.88\%\tabularnewline \hline
%      3 &14.26\%& 58.09\%& 12.27\%& 4.50\%& 7.19\%& 2.75\%& 0.94\%\tabularnewline \hline
%      4 &13.94\%& 58.11\%& 12.14\%& 4.78\%& 7.35\%& 2.68\%& 0.99\%\tabularnewline \hline
%      5 &13.92\%& 58.13\%& 11.93\%& 4.56\%& 7.60\%& 2.98\%& 0.88\%\tabularnewline \hline
%    \end{tabular}
%  \end{center}
%  This table presents the allocations of the wealth in the IRA
%  and Keogh accounts in various asset classes.
%  Results from each set of imputed data are presented here.
%  The first column lists the number of the imputations,
%  and rest of the columns lists various allocations.
%  Entrees under each asset class show the percentage of investors
%  who have most of their IRA
%  and Keogh wealth invested in that particular asset class.
%  The asset class Diversified
%  includes stocks,
%  bonds,
%  and money market investments.
%  The asset class Others
%  include investments in various life insurance products,
%  annuities,
%  real estate, etc.
%  \medskip
%  \footnotesize SOURCE: Survey of Consumer Finances,
%  2001,
%  Federal Reserve Board,
%  USA.\par
%\end{table}
%  }

\begin{table}
    \caption{Presidents}
    \begin{center}
        \begin{tabular}{cl}
            \#& Name\\
            1&  George Washington\\
            2&  John Adams\\
            3&  Thomas Jefferson\\
        \end{tabular}
    \end{center}
\end{table}

\begin{table}
    \caption{Presidents with horizontal and vertical lines}
    \begin{center}
        \begin{tabular}{|c|l|}
            \hline
            \#& Name\\
            \hline
            1&  George Washington\\
            \hline
            2&  John Adams\\
            \hline
            3&  Thomas Jefferson\\
            \hline
        \end{tabular}
    \end{center}
\end{table}

%\include{demo-text}

% Bibliography is required if you consulted any outside references.
% Reference: TM 32.
\include{bibliography}

% Notes and footnotes are optional.
% Reference: TM 34.
% I have not implemented this yet.  Mark Senn 2002-06-03
%%\include{notes}

% A vita is optional for masters theses
% and required for doctoral dissertations.
% Reference: TM 13.
% CHANGE NEXT LINE?
%\include{vita}

\end{document}

% LaTeX won't read after the \end{document} command.
% You can put notes to yourself or LaTeX input not
% ready for use here if you'd like.
