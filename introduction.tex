%
%  revised  introduction.tex  2011-09-02  Mark Senn  http://engineering.purdue.edu/~mark
%  created  introduction.tex  2002-06-03  Mark Senn  http://engineering.purdue.edu/~mark
%
%  This is the introduction chapter for a simple, example thesis.
%

\chapter{Introduction}
\label{chapter:intro}

There are many different computer systems in the world today. Many of these systems are general purpose computing
systems, such as consumer desktops and laptops. However, there are many more systems with a very specific use
and that interact with the world in a physical way. Examples of this include sensor arrays, surveillance systems,
or utility pipelines. These are called "physical systems".
These systems incorporate a large amount of computation to perform their jobs, but their main tasks accomplished
by interacting with the physical world in some way. 

These physical systems are beginning to become more and more complex. Originally, these systems computational
abilities were very limited, maybe being restricted to a few hard coded operations, potentially only accessible by
an on site technician. Today, many of these systems have a much greater computing capacity, have more
dynamic capabilities, and, especially, are more connected to some sort of network, such as the Internet.
These improvements have allowed for much greater control over systems, better remote interfacing, and greater
efficiency.

With all the improvements however also comes security risks. Suddenly, physical systems are vulnerable to malicious
control from an adversary connected over the internet. Besides just a networked adversary, it is possible that
malicious code, such as a virus, might infect the system. Due to the increased processing power and more 
generalized computing resources, these viruses would have a greater attack surface and more opportunities to
compromise such a system.

One of the main problems facing physical systems is that they have difficulty being identified uniquely. That is,
when someone is communicating remotely, they  are not completely sure that the system they are communicating
with is authentic. An attacker might have made a copy of the device and could be impersonating the device. 
Alternatively, the device could be a counterfeit. 
What is needed is some way to ensure that the device is actually the intended device.

A novel technology called Physically Unclonable Function (PUF) provides the sort of device identification that is needed
to solve the previous issue. A PUF is a device that can be used to generate a response that is unique to a given device.
PUFs are made by leveraging small inconsistencies in the manufacturing process. As such, it is impossible to
duplicate a PUF. Since the PUF cannot be duplicated, if a device ever returns the expected response from its PUF,
the other party can be confident that the device is the intended device.

% Discuss the structure of the thesis
The rest of the thesis is structured as followed. 
Chapter ~\ref{chapter:physicalsystems} describes physical systems and their nature, including several of the difficulties that are involved with them, in more depth. 

Chapter ~\ref{chapter:cryptographyoverview} provides some of the necessary cryptography background needed for
understanding the rest of the paper.

Chapter ~\ref{chapter:pufoverview} introduces PUF technology and creates an initial connection to physical systems. 
Several different PUF architectures are presented as well as a discussion of some implementation issues.

Chapters~\ref{chapter:rok},~\ref{chapter:pear}, and~\ref{chapter:doe} each describe an applications of PUF 
technology as it incorporated into physical systems. These applications demonstrate the use of PUF as a way of 
resolving the issues facing physical systems. 

Chapter~\ref{chapter:rok} describes a project called Read Once Keys. These are keys that once being read are
destroyed and are irrecoverable. The PUF device is used in this case as a way of providing trusted execution.

Chapter~\ref{chapter:pear} describes a project called Physically Enhanced Authentication Ring. This project uses
a PUF to combat a potentially compromised communication channel, such as when a key logger is installed.

Chapter~\ref{chapter:doe} describes a project regarding the smart grid and smart meters. A PUF is incorporated
into a smart meter and is used to securely authenticate with a utility company, in spite of potential threats.

Chapter ~\ref{chapter:conclusion} draws final conclusions and presents closing thoughts.
