\chapter{Cryptography Overview}
\label{chapter:cryptographyoverview}

% Provide a brief overview
\section{Overview}
Before delving into the details of various applications, it is necessary to define and understand several
cryptographic primitives, as they form a foundation to build off. The following sections present a brief
introduction to the necessary cryptographic primitives that will be used in the rest of the thesis.

% Encryption operations
\section{Encryption}
It is often necessary to scramble and protect data so that only certain parties, such as those who
possess a key value, can de-scramble and read the protected data. This might be necessary when
sending any sort of sensitive data, such as financial records or e-mail messages. Presumably, if a 
person does not have the correct key value, he or she will not be able to scramble or unscramble
the data properly.

The act of scrambling the data is called \emph{encryption}. The corresponding act of descrambling
encrypted data is callled \emph{decryption}.

Encryption and decryption operations and relevant parameters are denoted using the following notation below.

\begin{align}
C  = E_K(M) \\
M = D_K(C)
\end{align}

Above, C is the \emph{ciphertext}, or encrypted text.  
M is the message or \emph{plaintext}. 
K represents the \emph{key value}. 
E represents the \emph{encryption algorithm}, of which there are several types. This algorithm takes plaintext as a parameter and returns ciphertext.
D represents the \emph{decryption algorithm}, which takes ciphertext and return plaintext.

A sender would use his plaintext message to generate the ciphertext and transmit it. The receiver would then process the received data using the decryption
algorithm and then be able to successfully recover the plaintext message.

% Symmetric key encryption
\subsection{Symmetric Encryption}
Symmetric encryption is a fairly intuitive method of using encryption. In this style of encryption,
both the sender and receiver share the same key value, K.

% Asymmetric key encryption
\subsection{Asymmetric Encryption}

% ZKPK
\section{Zero Knowledge Proof of Knowledge}

% Committments
\section{Commitment Schemes}
